\documentclass[a4paper,10pt]{article}
\usepackage{fullpage}

\usepackage{amsmath,amssymb,amsthm}
\usepackage{stmaryrd}
\usepackage{mathtools}
\usepackage{epsfig}
\usepackage{hyperref}
\usepackage{epstopdf}
\usepackage{color}
\usepackage{colortbl}
\usepackage{multirow}% http://ctan.org/pkg/multirow
\usepackage{hhline}%
\usepackage{lscape}
\usepackage{graphicx}
\usepackage[inline]{enumitem}
\usepackage{xcolor}
\usepackage{array}
\usepackage{float}
\usepackage{tikz}
\usepackage{tikz-cd}
\usepackage{cleveref}
\usepackage{url}
\usepackage[makeroom]{cancel}

\usepackage{amsthm}
\usepackage{slashed}
\usepackage{tikz}
\usetikzlibrary{arrows.meta,graphs,quotes,decorations,intersections,calc}
\usetikzlibrary{fit}
\usepackage{tikzpeople}

\usepackage[
lambda,advantage,adversary,probability,
sets,notions,asymptotics,keys,
operators,ff,primitives
]{cryptocode}

\newcolumntype{C}[1]{>{\centering\arraybackslash}m{#1}}

\usepackage{tcolorbox}
\newcommand{\say}[1]{\emph{``{#1}''}}

\usepackage{listings}
\usepackage{color}

\definecolor{mygreen}{rgb}{0,0.6,0}
\definecolor{mygray}{rgb}{0.5,0.5,0.5}
\definecolor{mymauve}{rgb}{0.58,0,0.82}

\lstset{ 
  backgroundcolor=\color{white},   % choose the background color; you must add \usepackage{color} or \usepackage{xcolor}; should come as last argument
  basicstyle=\footnotesize,        % the size of the fonts that are used for the code
  breakatwhitespace=false,         % sets if automatic breaks should only happen at whitespace
  breaklines=true,                 % sets automatic line breaking
  captionpos=b,                    % sets the caption-position to bottom
  commentstyle=\color{mygreen},    % comment style
  deletekeywords={...},            % if you want to delete keywords from the given language
  escapeinside={\%*}{*)},          % if you want to add LaTeX within your code
  extendedchars=true,              % lets you use non-ASCII characters; for 8-bits encodings only, does not work with UTF-8
  firstnumber=1,                   % start line enumeration with line 1
  frame=single,                    % adds a frame around the code
  keepspaces=true,                 % keeps spaces in text, useful for keeping indentation of code (possibly needs columns=flexible)
  keywordstyle=\color{blue},       % keyword style
  language=Octave,                 % the language of the code
  morekeywords={*,...},            % if you want to add more keywords to the set
  numbers=left,                    % where to put the line-numbers; possible values are (none, left, right)
  numbersep=5pt,                   % how far the line-numbers are from the code
  numberstyle=\tiny\color{mygray}, % the style that is used for the line-numbers
  rulecolor=\color{black},         % if not set, the frame-color may be changed on line-breaks within not-black text (e.g. comments (green here))
  showspaces=false,                % show spaces everywhere adding particular underscores; it overrides 'showstringspaces'
  showstringspaces=false,          % underline spaces within strings only
  showtabs=false,                  % show tabs within strings adding particular underscores
  stepnumber=2,                    % the step between two line-numbers. If it's 1, each line will be numbered
  stringstyle=\color{mymauve},     % string literal style
  tabsize=2,                     % sets default tabsize to 2 spaces
  title=\lstname                   % show the filename of files included with \lstinputlisting; also try caption instead of title
}


% !TeX root = ../report.tex

% General Macros
% and Notation

\newcommand{\msf}[1]{\ensuremath{\textsf{#1}}}

\NewDocumentCommand{\oneinputalg}{R(){Alg}m}{%
	\IfBlankTF{#2}{%
		\ensuremath{{#1}}
	}{%
		\ensuremath{{#1}(#2)}
	}%
}

\NewDocumentCommand{\twoinputalg}{R(){Alg}mom}{%
	\IfBlankTF{#4}{%
		\ensuremath{{#1}}
	}{%
		\IfValueTF{#3}{%
			\ensuremath{{#1}(#3,#4)}
		}{%
			\ensuremath{{#1}(#2,#4)}
		}%
	}%
}

\NewDocumentCommand{\threeinputalg}{R(){Alg}momom}{%
	\IfBlankTF{#6}{%
		\ensuremath{#1}
	}{%
		\IfValueTF{#5}{%
			\IfValueTF{#3}{%
				\ensuremath{{#1}(#3,#5,#6)}
			}{
				\ensuremath{{#1}(#2,#5,#6)}
			}%
		}{%
			\IfValueTF{#3}{%
				\ensuremath{{#1}(#3,#4,#6)}
			}{%
				\ensuremath{{#1}(#2,#4,#6)}
			}%
		}%
	}%
}

\NewDocumentCommand{\fourinputalg}{R(){Alg}momomom}{%
	\IfBlankTF{#8}{%
		\ensuremath{#1}
	}{%
		\IfValueTF{#7}{%
			\IfValueTF{#5}{%
				\IfValueTF{#3}{%
					\ensuremath{{#1}(#3,#5,#7,#8)}
				}{
					\ensuremath{{#1}(#2,#5,#7,#8)}
				}%
			}{%
				\IfValueTF{#3}{%
					\ensuremath{{#1}(#3,#4,#7,#8)}
				}{%
					\ensuremath{{#1}(#2,#4,#7,#8)}
				}%
			}%
		}{%
			\IfValueTF{#5}{%
				\IfValueTF{#3}{%
					\ensuremath{{#1}(#3,#5,#6,#8)}
				}{
					\ensuremath{{#1}(#2,#5,#6,#8)}
				}%
			}{%
				\IfValueTF{#3}{%
					\ensuremath{{#1}(#3,#4,#6,#8)}
				}{%
					\ensuremath{{#1}(#2,#4,#6,#8)}
				}%
			}%
		}%
	}%
}

\NewDocumentCommand{\optindex}{om}{%
	\IfValueTF{#1}{%
		\ensuremath{{#2}_{#1}}%
	}{%
		\ensuremath{{#2}}%
	}%
}

\newcommand{\red}[1]{\textcolor{red}{#1}}
\newcommand{\oran}[1]{\textcolor{orange}{#1}}
\newcommand{\blue}[1]{\textcolor{blue}{#1}}
\newcommand{\gray}[1]{\textcolor{gray}{#1}}


\newcommand{\ok}{\msf{ok!}}
\newcommand{\bad}{\msf{bad}}

\newcommand{\valid}{\msf{valid}}
\newcommand{\fail}{\msf{fail}}


\newcommand{\PoSW}{\msf{PoSW}}

\newcommand{\qprime}{\ensuremath{q}}
\newcommand{\qpprime}{\ensuremath{p}}
\newcommand{\gammaR}{\ensuremath{\gamma_{\Rring}}}



\NewDocumentCommand{\witness}{o}{\optindex[#1]{\omega}}
\RenewDocumentCommand{\proof}{o}{\optindex[#1]{\pi}}
\NewDocumentCommand{\transcript}{o}{\optindex[#1]{\tau}}


\renewcommand{\vec}[1]{\ensuremath{\textbf{#1}}}

\NewDocumentCommand{\vx}{o}{\optindex[#1]{\vec{x}}}
\NewDocumentCommand{\vy}{o}{\optindex[#1]{\vec{y}}}
\NewDocumentCommand{\vu}{o}{\optindex[#1]{\vec{u}}}

\NewDocumentCommand{\vA}{o}{\optindex[#1]{\vec{A}}}
\NewDocumentCommand{\vG}{o}{\optindex[#1]{\vec{G}}}
\NewDocumentCommand{\vGi}{o}{\optindex[#1]{\vec{G}}^{-1}}

\NewDocumentCommand{\Rring}{o}{\optindex[#1]{\mathcal{R}}}

\RenewDocumentCommand{\pp}{o}{\optindex[#1]{\msf{pp}}}


\NewDocumentCommand{\ffun}{om}{\oneinputalg(\optindex[#1]{\msf{f}}){#2}}

\NewDocumentCommand{\hashf}{om}{\oneinputalg(\optindex[#1]{\msf{H}}){#2}}


\newcommand{\init}{\ensuremath{\msf{Init}}}
\newcommand{\gen}{\ensuremath{\msf{Gen}}}
\renewcommand{\eval}{\ensuremath{\msf{Eval}}}
\newcommand{\prove}{\ensuremath{\msf{P}}}
\renewcommand{\verify}{\ensuremath{\msf{V}}}
\newcommand{\proveR}{\ensuremath{\overline{\msf{P}}}}
\newcommand{\verifyR}{\ensuremath{\overline{\msf{V}}}}

\newcommand{\pprime}{\ensuremath{p}}



\newcommand{\evaluate}{\ensuremath{\textsf{Eval}}}
\newcommand{\Poly}{\ensuremath{\textsf{Poly}}}

\renewcommand\adversary[1]{\ensuremath{\mathcal{#1}}}
\newcommand{\advP}{\adversary{P}}
\newcommand{\advV}{\adversary{V}}



\newcommand{\card}[1]{\ensuremath{\left\lvert #1 \right\rvert}}
\newcommand{\comp}[1]{\ensuremath{\left\lVert #1 \right\rVert}}
\renewcommand{\asymp}[1]{\ensuremath{\mathcal{O}\left(#1\right)}}

\makeatletter
\newcommand{\tpmod}[1]{\mkern 3mu({\operator@font mod}\mkern 6mu#1)}
\makeatother


\newcommand{\cbnote}[1]{\todo[inline,caption={},color=blue!20]{cB: #1}}
\newcommand{\draftnote}[1]{\todo[inline,caption={},color=green!20]{#1}}
\newcommand{\mindnote}[1]{\todo[inline,caption={},color=gray!10]{#1}}

\def\BibTeX{{\rm B\kern-.05em{\sc i\kern-.025em b}\kern-.08em
    T\kern-.1667em\lower.7ex\hbox{E}\kern-.125emX}}



\usepackage{todonotes}
\newcommand{\cbnote}[1]{\todo[inline,caption={},color=red!10]{\footnotesize{}cB: #1}}
\newcommand{\metanote}[1]{\todo[inline,caption={},color=gray!10]{\footnotesize{}#1}}
\newcommand{\probnote}[1]{\todo[inline,caption={},color=blue!10]{\footnotesize{}#1}}

\author{Carlo Brunetta}
\title{Technical Assessment}

\begin{document}

\maketitle


\subsection{Document Structure}
Personal thoughts/solving processes are highlighted into grey boxes such as,
\metanote{
  this one! This is where I add mental notes.\\
  The notes might not be precise or correct but they are part of my solving process.
}
\noindent
The document will follow the effective analysis timing.

The problems are iteratively solved following the mantra: \emph{``read, take notes and repeat''}, thus trying to trace a solution shape as quickly as possible.
To achieve this, missing steps/problems will be highlighted for future analysis and will be highlighted
in blue boxes such as,
\probnote{Read the assignments and solve them!}


% ---


\section{Multi-Party Exponentiation}

\probnote{
  To solve the problem:
  \begin{enumerate}
    \item Read the paper \cite{FC:AAS18} and protocols (both for notation and context of the protocol).
    \item Point out what must be proven.
    \item Prove it!
    \item Check the code and pen-paper design of the proof of concept (PoC) implementation.
    \item Implement it
  \end{enumerate}
}

\subsection{Proving Security}

Denote the protocols of interest, Protocol 5 and Protocol 6, as $\proto{5},\proto{6}$ of which security
is defined in the UC framework.
The paper assumes the existence of an ideal \emph{arithmetic black box} functionality $\Fabb$, defined
and instantiated by Damg\aa{}rd et al.~\cite{TCC:DFKNT06}.
Assuming the latter is proven secure in the UC framework, to prove $\proto{5},\proto{6}$ UC-secure 
one must show that the exponentiation protocol $\pexp(b,\shares{a})$ required in $\proto{5},\proto{6}$,
not provided by $\Fabb$, is UC-secure.

The authors report a UC-secure protocol \proto{2}~\cite{TCC:DFKNT06} of which efficiency depends on the
exponent bit-length.
%
They provide a more efficient (independent on input bit-length, dependent on the amount of parties)
protocol \proto{4} which is pointed out to be insecure against active adversaries.
This protocol is utilized in \proto{5} and \proto{6}: briefly, \proto{4}\ is executed
twice with different inputs, related by an introduced randomness, and a final reconstruction verifies
that the protocol computation was indeed correctly executed.

\probnote{Check what the suggested active attack on \proto{4}\ does and if this can affect
  $\proto{5}$ or $\proto{6}$.}
\metanote{Composition of non UC-secure protocols (e.g. $\proto{4}$) is rarely (almost never) UC-secure.\\
  A quick skim of \cite{TCC:DFKNT06} shows the inexistence of \proto{2} as described.\\
  Maybe the authors~\cite{FC:AAS18} considered the exponentiation in the instantiation as part of the
  functionality protocol?
  Damg\aa{}rd et al. instantiate \Fabb\ via Paillier cryptosystem which effectively encrypts by
  computing several exponentiations \textbf{but} this does not represent \Fabb's operations.}

To show UC-security of \proto{4} (\Cref{fig:proto4}), one should find a simulator $\simul$ able to
emulate the ideal functionality $\Fexp$ with the execution of $\proto{4}$, even in the presence of
any adversary $\advA$.
The transcript is later provided to an environment $\advR$ which must distinguish between ideal
or emulated execution.

\subsection{Insecurity of $\proto{4}$}

Observe that the environment $\advR$ can instantiate $\advA$ with an input $\Delta$ not
known to the simulator $\simul$.
This is key to the active attack on \proto{4}, as highlighted in \Cref{fig:proto4}.

\begin{figure}[!ht]
    \centering
    \begin{pchstack}
      \procedure{Protocol $\proto{4}$: $\pexp(b,\shares{a})$}{
        \forall_{P_i}:\: c_i \gets b^{\alpha_i \cdot a_i}\\
        \shares{c_i} \gets \pshare\left( c_i \right)\\
        \shares{b^a} \gets \pproduct\left( \prod_{i=1}^n \shares{c_i} \right)
      }
      \pchspace
      \procedure{Adversarial $P_1=\advA(\Delta)$ execution of $\proto{4}$}{
        \pccomment{Instantiated $\advA$ with uniform random $\Delta \neq 1$}\\
        c_1 \gets b^{\alpha_1 \cdot a_1}\red{\cdot \Delta}\\
        \shares{c_1\red{\cdot \Delta}} \gets \pshare\left( c_1 \red{\cdot \Delta} \right)\\
        \shares{b^a \red{\cdot{\Delta}}} \gets \pproduct\left( \shares{c_1\red{\cdot \Delta}}
          \cdot \prod_{i=2}^n \shares{c_i} \right)
      }
    \end{pchstack}
    \caption{Protocol $\proto{4}$ and $\advA$'s malicious execution (highlighted in \red{red})
      as, w.l.o.g., party $P_1$.}\label{fig:proto4}
\end{figure}


Allowing the party to independently compute $b^{\alpha_i \cdot a_i}$ opens the adversary
to arbitrarily modify its share to any value; more specifically, we consider
$b^{\alpha_i \cdot a_i} \red{\cdot \Delta}$ where $\Delta \neq 1$ and provided by the environment during
the initialization phase.
When reconstructing the shares, the output will be $b^{a} \red{\cdot \Delta}$ thus corrupted.

\subsection{Insecurity of $\proto{6}$}

Let us report \proto{6}\ in \Cref{fig:proto6} and highlight the attack.
$\advA$ participates in the $\pexp$ executions by maliciously injecting the instantiation
coefficient $\Delta_1 = \Delta$ and later injects $\Delta_2 = \Delta^{-1}$.

\begin{figure}[!ht]
    \centering
    \begin{pchstack}
      \procedure{Protocol $\proto{6}$: $\pexp(b,\shares{a})$}{
        \shares{b^a\red{\Delta_1}} \gets \pexp(b,\shares{a})\\
        \shares{r} \gets \psrand(\ZZ_p^*)\\
        \shares{a^\prime} \gets \pproduct(\shares{a},\shares{r}-\shares{1})\\
        \shares{b^{a^\prime}\red{\Delta_2}} \gets \pexp(b,\shares{a^\prime})\\
        \shares{w} \gets \pproduct(\shares{a},\shares{r})\\
        w \gets \popen(\shares{w})\\
        \shares{r^\prime} \gets \psrand(\ZZ_q^*)\\
        %
        \shares{v} \gets \pproduct(\shares{r^\prime},
          \pproduct[\pproduct(\shares{b^a\red{\Delta_1}},\shares{b^{a^\prime}\red{\Delta_2}}), b^{-w}]
          - \shares{1}) + \shares{1}\\
        v \gets \popen(\shares{v})
      }
    \end{pchstack}
    \caption{Protocol $\proto{6}$. Highlighted in \red{red} the weak exponential
      executions.}\label{fig:proto6}
\end{figure}

The injected coefficients cancel out and allow the correct opening of $v$ to $1$; however the
exponentiation shares opens to the incorrect value $b^{a}\red{\cdot \Delta}$.
A simulator $\simul$ is able to emulate the attack, however it has $\frac{1}{q-1}$ probability of
correctly guessing $\Delta$; thus, the environment has a high probability of distinguishing between
real execution and ideal functionality.
Therefore, 

\begin{proposition}
  Assuming UC-secure $\Fabb$ exists, consider the protocol $\proto{4}$
  (\Cref{fig:proto4}), then $\proto{6}$ is UC-insecure.
\end{proposition}

From my quick search, $\proto{2}$ is not defined by Damg\aa{}rd et al.~\cite{TCC:DFKNT06}, thus it is
not possible to check if it would allow $\proto{6}$ to be UC-secure.
\probnote{Future development, search/design a UC-secure instantiation of $\pexp(b,\shares{a})$.}


\subsection{Insecurity of \proto{5}}
\metanote{While writing about the attack on $\proto{6}$, I noticed that $\proto{5}$ utilizes the other
  exponentiation protocol $\pexp(\shares{b},a)$ (public exponent, private base).
  This required to check the provided $\proto{7}$ since $\Fabb$ does not provide ideal functionality
  and Damg\aa{}rd et al.'s reported protocol $\proto{1}$ is not defined in the original
  paper~\cite{TCC:DFKNT06} (similarly as before).}

Regarding security of $\proto{5}$, observe that the verification code must execute the public
exponent exponentiation $\pexp(\shares{b^a},r)$ of which instantiation is provided by protocols
$\proto{1}$ and $\proto{7}$, reported in \Cref{fig:proto57}.
\probnote{Double-check UC-security and definition of Damg\aa{}rd et al.'s in \proto{1}~\cite{TCC:DFKNT06}}
\probnote{Check if $\proto{7}$ is secure and, if not, how this can be used to attack $\proto{5}$}
%
Similarly to before, the adversary injects $\Delta$ in the first call of \proto{5}, i.e.
$\shares{b^a \Delta_1}\gets \pexp(b,\shares{a})$, and later injects $\Delta^{-1}$ in the
call $\shares{r^\prime\cdot \Delta^{-1}} \gets \pexp(g, \shares{r})$.
This attack forces the exponentiation $\pexp(\shares{b^a \Delta},r)$ to output $\shares{b^{ar}}$
which correctly verifies the computation (i.e. the protocol outputs $v=1$); however, the shares open
to $b^a \Delta$ and, as before, allow the environment to distinguish between real and ideal
execution.

\begin{figure}[!ht]
    \centering
    \begin{pchstack}
      \procedure{Protocol $\proto{5}$: $\pexp(b,\shares{a})$}{
        \shares{b^a \red{\cdot \Delta}} \gets \pexp(b,\shares{a})\\
        \shares{r} \gets \psrand(\ZZ_p^*)\\
        \shares{a^\prime} \gets \pproduct(\shares{a},\shares{r})\\
        \shares{b^{a^\prime}} \gets \pexp(b,\shares{a^\prime})\\
        r \gets \popen(\shares{r})\\
        \shares{r^\prime} \gets \psrand(\ZZ_q^*)\\
        \shares{v} \gets \pproduct
          \left[\shares{r^\prime},\pexp(\shares{b^a\red{\cdot \Delta}},r) - \shares{b^{a^\prime}}\right]
          + \shares{1}\\
        v \gets \popen(\shares{v})
      }
      \pchspace
      \procedure{Protocol $\proto{7}$: $\pexp(\shares{b\red{\cdot \Delta}},a)$}{
        g \gets \sffont{getGenerator}\\
        \shares{r} \gets \psrand(\ZZ_p^*)\\
        \shares{r^\prime\red{\cdot \Delta^{-1}}} \gets \pexp(g,\shares{r})\\
        \shares{c} \gets \pproduct(\shares{r^\prime\red{\cdot \Delta^{-1}}},\shares{b\red{\cdot \Delta}})\\
        c \gets \popen(\shares{c})\\
        c^\prime \gets c^a\\
        \shares{e} \gets -a \cdot \shares{r}\\
        \shares{b^a} \gets c^\prime \cdot \pexp(g,\shares{e})
      }
    \end{pchstack}
    \caption{Protocol $\proto{5}$ and $\proto{7}$.
      Highlighted in \text{red} the attack.}\label{fig:proto57}
\end{figure}


\begin{proposition}
  Assume UC-secure $\Fabb$ exists, consider the protocol $\proto{7}$~\cite{FC:AAS18},
  then $\proto{5}$ is UC-insecure.
\end{proposition}

\metanote{There might be an easier attack similar to the \proto{6}'s one by doing some
  injection in the exponent $a+\delta$ but it would require some math to check where to
  inject $\Delta^{-1}$.}


% ---

\subsection{Proof of Concept Implementation}

\metanote{Before implementing the code, I designed the code pen-and-paper
  with the goal of highlighting how (Shamir's) secret sharing is effectively incorporated
  in the protocol.}

The code is designed in several interconnected classes, as shown in \Cref{fig:expcode}, which can be
described as:
\begin{description}
  \item[Finite Field] implements the finite field operations used by the other classes.
  \item[Share] represents the secret share from the secret-sharing algorithm.
  \item[Point] represents the evaluation point obtained from the secret-sharing algorithm and
    used to identify the party and reconstruct the secret shared.
  \item[Party] represents the protocol's party with its shares, points and
    that collaborates in the protocol execution (via $\Fabb$ and $\pexp$).
    The party has a corruption functionality that stores $\Delta$.
    The $\pexp$ function can be set to be used in the attack.
  \item[Secret Sharing] implements (Shamir's) secret sharing algorithm.
  \item[Ideal $\Fabb$] implements the ideal functionality $\Fabb$ and has reference to all
    the parties in the protocol.
  \item[$\pexp(b,\shares{a})$] implements the protocols $\proto{4},\proto{5},
    \proto{6},\proto{7}$ and a PoC of the attack to $\proto{6}$.
\end{description}


\begin{figure}[!ht]
  \centering
  % To allow compilation of the file
% !TeX root = ./../reply.tex

\begin{tikzpicture}
  \draw (0,0) node [scale=1.0]{
    \begin{tikzpicture}
      [
        txt/.style={sloped,font=\scriptsize,pos=.5},
        txtv/.style={font=\scriptsize},
        construction/.style={color=green,thick},
        notion/.style={color=blue,thick},
        impossible/.style={color=red, thick},
        trivial/.style={color=black,thick,double},
        conjecture/.style={color=orange,thick,decorate,decoration={snake,amplitude=1}},
        related/.style={dashed,thick}
      ]

      \def\x{2};     % xshift constant
      \def\y{1};     % yshift constant

      \draw (0,\y*1+0.25) node (fg) {\TZrect{blue!10}{\x}{\y}{Finite Field}};
      \draw (0,\y*0) node (share) {\TZrect{blue!10}{\x}{\y}{Shares}};
      \draw (0,\y*-1-0.25) node (point) {\TZrect{blue!10}{\x}{\y}{Point}};
      
      \draw (1.5*\x,\y) node (ss) {\TZrect{yellow!10}{\x}{1.5*\y}{Secret Sharing}};
      \draw (1.5*\x,-\y) node (party) {\TZrect{yellow!10}{\x}{1.5*\y}{Party}};
      
      \draw (3*\x,\y*0) node (ideal) {\TZrect{green!10}{\x}{2.5*\y}{Ideal $\Fabb$}};
      
      \draw (4.5*\x,\y*0) node (exp) {\TZrect{green!10}{\x}{2.5*\y}{$\pexp(b,\shares{a})$}};
      
      \draw[-Stealth,trivial] (fg) to (ss);
      \draw[-Stealth,trivial] (share) to (party);
      \draw[-Stealth,trivial] (point) to (party);
      
      \draw[-Stealth,trivial] (ss) to (ideal);
      \draw[-Stealth,trivial] (party) to (ideal);
      
      \draw[Stealth-Stealth,trivial] (ideal) to (exp);
      
    \end{tikzpicture}
  };
\end{tikzpicture}

  \caption{Abstract representation of the implementation subdivided into classes.}\label{fig:expcode}
\end{figure}

To run the code, simply run:
\begin{lstlisting}[language=bash]
  $ python code/exp/main.py
\end{lstlisting}


Several warnings will appear which relate to specific implementation assumptions:
\begin{itemize}
  \item There is no input sanitization, i.e. there are no checks nor errors, no handling
    for wrong parameters $n,t,p$ or the sampled objects $\Delta, \alpha_i,r_i$.
    Forced \emph{always secure} intervals are used.
  \item The \Fabb\ handles the whole protocol without simulating the effective parties communication.
  \item $\pi_4,\pi_7$ only work for $n=t$.
    This is because the parties must compute $b^{\alpha_i\cdot a_i}$ where $\alpha_i$ is the
    Lagrange coefficient to be used for the reconstruction.
    This coefficient depends on the parties used in the final reconstruction which modifies the
    coefficients and the effective protocol execution (which should be limited to the $t$ 
    parties that will reconstruct the final result).
  \item Shares of exponents and finite field elements live in different groups.
    The paper itself discusses strategies to do this transformation and points to a specific paper.
    The PoC code naively opens and shares in the correct space leaving the correct handling for
    future development.
  \item The code only contains a showcase of $\proto{6}$'s attack.
    The attack to $\proto{5}$ can be similarly implemented.
\end{itemize}

\probnote{
  Further developments: 
  \begin{itemize}
    \item clean up the code and better handle the intrinsic difference between shares of exponents
      and elements of the finite field.
    \item find/define a secure and efficient protocol for the exponentiation (substitute $\pi_4$)
    \item for completeness, implement attack to $\proto{5}$
  \end{itemize}
}


% ---
\clearpage
% ---

\section{Zero Knowledge Proof}

\metanote{Read and report the $\Sigma$ protocol.
  Design the implementation and interactions between the classes.}

Consider a finite group $(\GG,\circ)$ with generator $g$ and order $p$ prime.
Observe that any element $x \in \GG$ has a unique coefficient $a \in \ZZ_p$ such that
$\underbracket{g \circ \dots \circ g}_{a \text{ times}} = x$ briefly denoted $\circ^a g$ and that
can be used to represent the scalar product of $a$ times of $g$.
\Cref{fig:chaped} reports Chaum–Pedersen $\Sigma$-protocol~\cite{BOOK:Smart03} described in a generic
finite group $(\GG,\circ)$ or order $p$ prime and with $g,h$ generators of $\GG$ and $y_1,y_2 \in \GG$.
The protocol's goal is proving the knowledge of a coefficient $x$ such that
$y_1 = \circ^x g$ and $y_2 = \circ^x h$.

\begin{figure}[!ht]
    \centering
    \begin{pchstack}
    
      \procedure{Prover's $\pcommit()$}{
        r \sample \ZZ_p^*\\
        (r_1,r_2) \gets (\circ^r g , \circ^r h)\\
        \pcreturn (r_1,r_2)
      }
      \pchspace
      \procedure{Verifier's $\pchal()$}{
        c \sample \{0,1\}\\
        \pcreturn c
      }
      \pchspace
      \procedure{Prover's $\popen(\red{r},\red{x},c)$}{
        s \gets r - c\cdot x\\
        \pcreturn s
      }
      \pchspace
      \procedure{$\pverify(y_1,y_2,r_1,r_2,c,s)$}{
        r_1 \overset{?}{=} (\circ^s g) \circ (\circ^c y_1)\\
        r_2 \overset{?}{=} (\circ^s h) \circ (\circ^c y_2)\\
        \pcreturn 1 \text{ if checks pass}\\
        \pcreturn 0 \text{ otherwise}
      }
    \end{pchstack}
    \caption{Chaum–Pedersen $\Sigma$-protocol represented as pseudo-code.
      In \red{red}, the prover's secret inputs.}\label{fig:chaped}
\end{figure}

Such an algorithm can be easily transformed into an authentication algorithm by noticing that
the protocol instantiation, i.e. the publication of the values $(y_1,y_2) = (g^{x_1},h^{x_2})$
with $x_1=x_2$, can be seen as the registration phase and the $\Sigma$-protocol as the authentication
phase (\Cref{fig:auth}).

\begin{figure}[!ht]
  \centering
  % To allow compilation of the file
% !TeX root = ./../reply.tex

\begin{tikzpicture}
  \draw (0,0) node [scale=1.0]{
    \begin{tikzpicture}
      [
        txt/.style={sloped,font=\scriptsize,pos=.5},
        txtv/.style={font=\scriptsize},
        construction/.style={color=green,thick},
        notion/.style={color=blue,thick},
        impossible/.style={color=red, thick},
        trivial/.style={color=black,thick,double},
        conjecture/.style={color=orange,thick,decorate,decoration={snake,amplitude=1}},
        related/.style={dashed,thick}
      ]

      \def\x{4};
      \def\y{0.5};

      \draw  (\x,\y) node[conductor,female,monitor,mirrored,minimum size=0.5cm] {\footnotesize Server};
      \draw  (-\x,\y) node[duck,minimum size=0.5cm] {\footnotesize User};
      
      \fill [orange!10,draw=black!20,rounded corners] (-\x-2,-4.5*\y) rectangle (\x+2,-0.25);
      \draw (\x+2.25,-2.5*\y) node[rotate=-90] {\scriptsize\textbf{Registration}};
      \draw (-\x,-\y) node[txtv] {\textbf{Password:} $\red{x} \in \ZZ_p^*$};
      \draw (\x,-2*\y) node[txtv] (g) {\textbf{Group:} $(\GG,\circ),g,h$};
      \draw[-Stealth] (g) -- ++ (-2*\x,0);
      \draw (-\x,-3*\y) node[txtv] (regU) {$(\circ^{\red{x}} g, \circ^{\red{x}} h)$};
      \draw (\x,-3*\y) node[txtv] (regS) {$(y_1,y_2)$};
      \draw (\x,-4*\y) node[txtv] {Store $(y_1,y_2)$ for User};
      \draw[-Stealth] (regU) -- (regS);
      
      \fill [green!10,draw=black!20,rounded corners] (-\x-2,-9.5*\y) rectangle (\x+2,-4.65*\y);
      \draw (\x+2.25,-7*\y) node[rotate=-90] {\scriptsize\textbf{Authentication}};
      \draw (-\x,-5*\y) node[txtv] {$(r_1,r_2) \gets \pcommit()$, $\red{r}$};
      \draw (-\x,-6*\y) node[txtv] (com) {$(r_1,r_2)$};
      \draw[-Stealth] (com) -- ++ (2*\x,0);
      \draw (\x,-7*\y) node[txtv] (cha) {$c \gets \pchal()$};
      \draw[-Stealth] (cha) -- ++ (-2*\x,0);
      \draw (-\x,-8*\y) node[txtv] (ope) {$s \gets \popen(\red{r},\red{x},c)$};
      \draw[-Stealth] (ope) -- ++ (2*\x,0);
      \draw (\x,-9*\y) node[txtv] {$\pverify(y_1,y_2,r_1,r_2,c,s)$};

    \end{tikzpicture}
  };
\end{tikzpicture}

  \caption{1-factor authentication protocol.}\label{fig:auth}
\end{figure}


% ---

\subsection{Implementation}

The code is designed in several interconnected classes, as shown in \Cref{fig:authcode}, which can be
described as:
\begin{description}
  \item[Finite Group] abstracts the group operations required by the $\Sigma$-protocol
    and allows the implementations to the specific instances.
  \item[ECC] represents an elliptic curve group, instance of finite group.
  \item[$\ZZ_p$] represents a modulo group of order $p$, instance of finite group.
  \item[CP $\Sigma$] implements Chaum-Pedersen $\Sigma$-protocol.
  \item[User] represents the user functionalities and stores the secret input.
  \item[Server] represents the server functionalities.
  \item[Auth] implements the protocol by handling the user and server's communication.
\end{description}


\begin{figure}[!ht]
  \centering
  % To allow compilation of the file
% !TeX root = ./../reply.tex

\begin{tikzpicture}
  \draw (0,0) node [scale=1.0]{
    \begin{tikzpicture}
      [
        txt/.style={sloped,font=\scriptsize,pos=.5},
        txtv/.style={font=\scriptsize},
        construction/.style={color=green,thick},
        notion/.style={color=blue,thick},
        impossible/.style={color=red, thick},
        trivial/.style={color=black,thick,double},
        conjecture/.style={color=orange,thick,decorate,decoration={snake,amplitude=1}},
        related/.style={dashed,thick}
      ]

      \def\x{2};    
      \def\y{1};    

      \draw (0,\y*0.65) node (fg) {\TZrect{blue!10}{\x}{\y}{ECC}};
      \draw (0,-\y*0.65) node (share) {\TZrect{blue!10}{\x}{\y}{$\ZZ_p$}};
      
      \draw (1.5*\x,\y*0) node (ss) {\TZrect{yellow!10}{\x}{1.5*\y}{Finite Group}};
      
      \draw (3*\x,\y*0) node (ideal) {\TZrect{green!10}{\x}{1.5*\y}{CP $\Sigma$}};
      
      \draw (4.5*\x,\y*0.65) node (user) {\TZrect{yellow!10}{\x}{\y}{User}};
      \draw (4.5*\x,-\y*0.65) node (server) {\TZrect{yellow!10}{\x}{\y}{Server}};
      
      \draw (6*\x,\y*0) node (exp) {\TZrect{green!10}{\x}{1.5*\y}{Auth Proto}};
      
      \draw[-Stealth,trivial] (fg) to (ss);
      \draw[-Stealth,trivial] (share) to (ss);
      
      \draw[-Stealth,trivial] (ss) to (ideal);
      
      \draw[Stealth-Stealth,trivial] (ideal) to (user);
      \draw[Stealth-Stealth,trivial] (ideal) to (server);
      
      \draw[Stealth-Stealth,trivial] (user) to (exp);
      \draw[Stealth-Stealth,trivial] (server) to (exp);
      
    \end{tikzpicture}
  };
\end{tikzpicture}

  \caption{Abstract representation of the authentication proof of concept.}\label{fig:authcode}
\end{figure}

The code will showcase one protocol execution in detail and execute a larger amount of
authentication and display the number of failing attempts.
%
To run the code, simply run (the Python package \textsf{tqdm} is required for the test
  progression bar):
\begin{lstlisting}[language=bash]
  $ python code/auth/main.py
\end{lstlisting}

For the sake of the PoC, small groups are chosen and fixed.
A note is left when using the implementation to highlight that the code is a PoC.
Several limitations are:
\begin{itemize}
  \item There is no input sanitization, i.e. there are no checks nor errors, no handling
    for wrong parameters or the sampled objects.
    Forced \emph{always secure} intervals are used.
  \item The elliptic curve implementation is general, simplistic and not efficient, i.e. during
    instantiation the code finds a generator for the biggest prime subgroup in the manifold by
    brute force and it does not easily support known curves.
    
\end{itemize}

\probnote{
  Further developments: 
  \begin{itemize}
    \item clean up the code and merge with an interface for effective communication
    \item the elliptic curve code is a simplistic adaptation of the
      \href{https://gist.github.com/bellbind/1414867/03b4b2dd79b41e65e51716076e5e2b0171628a10}
      {publicly available code}
    \item for real application, usage of side-channel--secure implementations for finite groups
      is to be preferred and introduced
    \item for real application, the registration phase is critical and it would improve to ask
      an immediate authentication to verify the user indeed provided the correct inputs
    \item for real application, the authentication can be transformed into a non-interactive
      protocol via Fiat-Shamir transform (or similar)
  \end{itemize}
}

% ---

\bibliographystyle{alphaurl}
\bibliography{biblio}

\end{document} 