% To allow compilation of the file
% !TeX root = ./reply.tex


\newcommand{\proto}[1]{\ensuremath{\Pi_{#1}}}
\newcommand{\Fabb}{\ensuremath{\mathcal{F}_{\textsf{ABB}}}}
\newcommand{\Fexp}{\ensuremath{\mathcal{F}_{\pexp}}}
\newcommand{\shares}[1]{\ensuremath{\left[{#1}\right]}}
\newcommand{\pexp}{\ensuremath{\sffont{exp}}}

\newcommand{\pshare}{\ensuremath{\sffont{Share}}}
\newcommand{\psrand}{\ensuremath{\sffont{sRand}}}
\newcommand{\pproduct}{\ensuremath{\sffont{Product}}}
\newcommand{\popen}{\ensuremath{\sffont{Open}}}
\newcommand{\pcommit}{\ensuremath{\sffont{Commit}}}
\newcommand{\ppolyopen}{\ensuremath{\sffont{PolyOpen}}}
\newcommand{\pchal}{\ensuremath{\sffont{Challenge}}}
\newcommand{\pverify}{\ensuremath{\sffont{Verify}}}
\newcommand{\ppolyverify}{\ensuremath{\sffont{PolyVerify}}}
\newcommand{\pinit}{\ensuremath{\sffont{Init}}}

\newcommand{\sffont}[1]{\ensuremath{\textsf{#1}}}


\newcommand{\red}[1]{\textcolor{red}{#1}}
\newcommand{\oran}[1]{\textcolor{orange}{#1}}
\newcommand{\gray}[1]{\textcolor{gray}{#1}}


\newcommand{\ok}{\sffont{\;ok!}}
\newcommand{\nok}{\sffont{\;not ok!}}
\newcommand{\bad}{\sffont{\;bad}}

\newcommand{\strat}[1]{\ensuremath{\sffont{S}_{#1}}}

\newcommand{\SF}{\sffont{SF}}
\newcommand{\PKE}{\sffont{PKE}}

\newcommand{\IND}{\sffont{IND}}
\newcommand{\SIND}{\sffont{$\Sigma$}\IND}
\newcommand{\DIND}{\sffont{D}\IND}

\newcommand{\LOR}{\sffont{LOR}}
\newcommand{\AND}{\sffont{AND}}
\newcommand{\KR}{\sffont{KR}}
\newcommand{\UKU}{\sffont{UKU}}
\newcommand{\ROR}{\sffont{ROR}}

\newcommand{\CXA}{\sffont{CXA}}
\newcommand{\CPA}{\sffont{CPA}}
\newcommand{\CCA}{\sffont{CCA}}

\newcommand{\eH}{\sffont{H}}
\newcommand{\gW}{\sffont{W}}
\newcommand{\pgW}[1]{\ensuremath{{\omega}_{#1}}}


\newcommand{\ent}[1]{\ensuremath{\mathsf{H}\left({#1}\right)}}
\newcommand{\cent}[2]{\ensuremath{\mathsf{H}\left({#1}\middle\lvert{#2}\right)}}
\newcommand{\dent}[1]{\ensuremath{\mathsf{h}\left({#1}\right)}}
\newcommand{\cdent}[2]{\ensuremath{\mathsf{h}\left({#1}\middle\lvert{#2}\right)}}

\newcommand{\mutinf}[2]{\ensuremath{\mathsf{I}\left({#1}\;;\;{#2}\right)}}
\newcommand{\mch}[2]{
\left.\mathchoice
  {\left(\kern-0.48em\binom{#1}{#2}\kern-0.48em\right)}
  {\big(\kern-0.30em\binom{\smash{#1}}{\smash{#2}}\kern-0.30em\big)}
  {\left(\kern-0.30em\binom{\smash{#1}}{\smash{#2}}\kern-0.30em\right)}
  {\left(\kern-0.30em\binom{\smash{#1}}{\smash{#2}}\kern-0.30em\right)}
\right.}


%
% ZK Stuff
%

\newcommand{\rs}[3]{\ensuremath{\sffont{RS}\left[{#1},{#2},{#3}\right]}}
\newcommand{\reval}[2]{\ensuremath{\left.{#1}\right\rvert_{#2}}}
\newcommand{\interpolant}[1]{\ensuremath{\sffont{iP}^{#1}}}
\newcommand{\polydeg}[1]{\ensuremath{\partial\left({#1}\right)}}
\newcommand{\zeropoly}[1]{\ensuremath{\sffont{Z}_{#1}}}

\newcommand{\comm}[1]{\ensuremath{\sffont{c}_{#1}}}
\newcommand{\proo}[1]{\ensuremath{\pi_{#1}}}

\newcommand{\redshift}{\ensuremath{\sffont{RedShift}}}
\newcommand{\plonkyy}{\ensuremath{\sffont{Plonky2}}}
\newcommand{\FRI}{\ensuremath{\sffont{FRI}}}
\newcommand{\DEEP}{\ensuremath{\sffont{DEEP}}}
\newcommand{\DEEPFRI}{\ensuremath{\sffont{\DEEP--\FRI}}}
\newcommand{\ethSTARK}{\ensuremath{\sffont{ethSTARK}}}

\newcommand{\hamdist}[2]{\ensuremath{\Delta\left({#1},{#2}\right)}}


%
% Random Math Notation / Functions
%

\newcommand{\range}{\ensuremath{\textrm{Rg}}}
\renewcommand*{\argmin}{\ensuremath{\text{arg\,min}}}
\newcommand{\pr}[1]{\ensuremath{\mathsf{Pr}\left[{#1}\right]}}
\newcommand{\cpr}[2]{\ensuremath{\mathsf{Pr}\left[{#1}\mid{#2}\right]}}
\newcommand{\complexity}[1]{\ensuremath{\mathcal{O}\left({#1}\right)}}
\newcommand{\pprime}{\ensuremath{p}}
\newcommand{\definput}{\ensuremath{x}}
\newcommand{\defoutput}{\ensuremath{y}}
\newcommand{\mean}[1]{\ensuremath{\mathbb{E}\left[{#1}\right]}}
\newcommand{\var}[1]{\ensuremath{\mathsf{Var}\left[{#1}\right]}}
\newcommand{\stirling}[2]{\ensuremath{\genfrac{\{}{\}}{0pt}{}{#1}{#2}}}
\newcommand{\factn}[2]{\ensuremath{\left({#1}\right)_{#2}}}


%
% Any kind of Key
%

% Standard
% \newcommand{\pk}{\ensuremath{\textsf{pk}}}
% \newcommand{\sk}{\ensuremath{\textsf{sk}}}
% \newcommand{\vk}{\ensuremath{\textsf{vk}}}
\newcommand{\ek}{\ensuremath{\textsf{ek}}}
\newcommand{\gsk}{\ensuremath{\textsf{g}\sk}}

% Master
\newcommand{\mvk}{\ensuremath{\textsf{m}\vk}}
\newcommand{\msk}{\ensuremath{\textsf{m}\sk}}
\newcommand{\mpk}{\ensuremath{\textsf{m}\pk}}

% Generics
\newcommand{\tamp}[1]{\ensuremath{{#1}^\star}}
\newcommand{\tomp}[1]{\ensuremath{{#1}_\star}}
\newcommand{\msg}{\ensuremath{\textsf{m}}}
\newcommand{\msgg}{\ensuremath{\textsf{m}^\prime}}



\newcommand{\setup}{\ensuremath{\textsf{Setup}}}
\newcommand{\init}{\ensuremath{\textsf{Init}}}
\newcommand{\keygen}{\ensuremath{\textsf{KGen}}}
\newcommand{\compute}{\ensuremath{\textsf{Compute}}}
\newcommand{\evaluate}{\ensuremath{\textsf{Eval}}}
\newcommand{\Poly}{\ensuremath{\textsf{Poly}}}


\newcommand{\Enc}{\ensuremath{\textsf{Enc}}}
\newcommand{\Dec}{\ensuremath{\textsf{Dec}}}

\renewcommand\adversary[1]{\ensuremath{\mathcal{#1}}}
\newcommand{\oracle}{\ensuremath{\textsf{O}}}
\newcommand\oracleN[1]{\ensuremath{{\oracle}_{\textsf{#1}}}}

\newcommand{\simul}{\ensuremath{\mathcal{S}}}
\newcommand{\experiment}[3]{\ensuremath{\textsf{Exp}^{#1}_{#2}(#3)}}
\renewcommand{\advantage}[3]{\ensuremath{\textsf{Adv}^{#1}_{#2}(#3)}}

\newcommand{\advA}{\adversary{A}}
\newcommand{\advB}{\adversary{B}}
\newcommand{\advC}{\adversary{C}}
\newcommand{\advR}{\adversary{R}}


%
% Any kind of party
%

\newcommand{\party}[1]{\ensuremath{\mathcal{#1}}}
\newcommand{\client}{\party{U}}
\newcommand{\server}{\party{S}}
\newcommand{\third}{\party{T}}


%
% Any knd of GameName / SecModel
%

\newcommand{\unf}{\ensuremath{\textsf{UNF}}}
\newcommand{\staticunf}{\ensuremath{\unf}}

\newcommand{\Mspace}{\ensuremath{\mathcal{M}}}
\newcommand{\Cspace}{\ensuremath{\mathcal{C}}}
\newcommand{\Kspace}{\ensuremath{\mathcal{K}}}
\newcommand{\Lspace}{\ensuremath{\mathcal{L}}}
\newcommand{\LspaceN}[1]{\ensuremath{\Lspace_{#1}}}

\newcommand{\Xspace}{\ensuremath{\mathcal{X}}}
\newcommand{\Yspace}{\ensuremath{\mathcal{Y}}}

\newcommand{\ghash}{\ensuremath{\overline{G}}}
\newcommand{\dhash}{\ensuremath{\overline{\hash}}}
\newcommand{\khash}[1]{\ensuremath{{F}_{#1}}}



%
% Group
%

\newcommand{\geng}{\ensuremath{g}}
\newcommand{\gengn}[1]{\ensuremath{g_{#1}}}
\newcommand{\emapempty}{\ensuremath{\emap{\gengn{1}}{\gengn{2}}}}


% Any other useful command

\newcommand\randomin[1]{\ensuremath{\sample {#1} }}
\newcommand\valuein{\ensuremath{\in}}
\newcommand\bitstring[1]{\ensuremath{{\{0,1\}}^{{#1}} }}

\newcommand\lpnorm[2]{\ensuremath{\left\lVert #2 \right\rVert_{#1}}}
\newcommand\closint[1]{\ensuremath{\left\lfloor #1 \right\rceil}}

\newcommand{\card}[1]{\ensuremath{\left\lvert #1 \right\rvert}}

\newcommand{\qedb}{\hfill\ensuremath{\blacksquare}}

\makeatletter
\newcommand{\tpmod}[1]{\mkern 3mu({\operator@font mod}\mkern 6mu#1)}
\makeatother


\newcommand{\ie}{\emph{i.e.}}
\newcommand{\eg}{\emph{e.g.}}
\newcommand{\Lwlog}{\textit{w.l.o.g.}}
\newcommand{\wrt}{\textit{w.r.t.}}
\newcommand{\etal}{\emph{et al.}} 



\newcommand{\tikzmark}[2]{\tikz[overlay,remember picture,baseline=(#1.base)] \node (#1) {#2};}
%
\newcommand{\Highlight}[4][submatrix]{%
    \tikz[overlay,remember picture]{
    \node[rectangle,rounded corners,draw,fill opacity=0.2,inner sep=0pt,
    	fill=#2,fit=(#3.north west) (#4.south east)] (#1) {};}
}




% Math Environment

\newtheorem{theorem}{Thm.}{}
\newtheorem{definition}{Def.}{}
\newtheorem{proposition}{Prop.}{}
\newtheorem{corollary}{Cor.}{}
\newtheorem{lemma}{Lemma}{}
\newtheorem{claim}{Claim}{}
\newtheorem{remark}{Rem.}{}
\newtheorem{problem}{Prob.}{}

\theoremstyle{remark}
\newtheorem*{sketch}{Sketch}{}

\Crefname{theorem}{Thm.}{Thms.}
\Crefname{definition}{Def.}{Defs.}
\Crefname{proposition}{Prop.}{Props.}
\Crefname{corollary}{Cor.}{Cors.}
\Crefname{figure}{Fig.}{Figs.}


% Random Macros

\makeatletter
\renewcommand*\env@matrix[1][*\c@MaxMatrixCols c]{%
  \hskip -\arraycolsep
  \let\@ifnextchar\new@ifnextchar
  \array{#1}}
\makeatother


% TikZ

\newcommand{\TZrect}[4]{
  \begin{tikzpicture}
    \draw [thick,draw=black, fill=#1, rounded corners] (0,0) rectangle  ++ (#2,#3)
      node[text width=18mm,pos=.5,scale=0.9,align=center] {\textbf{{#4}}};
  \end{tikzpicture}}
