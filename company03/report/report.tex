\documentclass[a4paper,10pt]{article}
\usepackage{fullpage}

\usepackage{amsmath,amssymb,amsthm}
\usepackage{stmaryrd}
\usepackage{mathtools}
\usepackage{epsfig}
\usepackage{hyperref}
\usepackage{epstopdf}
\usepackage{color}
\usepackage{colortbl}
\usepackage{multirow}% http://ctan.org/pkg/multirow
\usepackage{hhline}%
\usepackage{lscape}
\usepackage{graphicx}
\usepackage[inline]{enumitem}
\usepackage{xcolor}
\usepackage{array}
\usepackage{float}
\usepackage{tikz}
\usepackage{tikz-cd}
\usepackage{cleveref}
\usepackage{url}
\usepackage[makeroom]{cancel}

\usepackage{amsthm}
\usepackage{slashed}
\usepackage{tikz}
\usetikzlibrary{arrows.meta,graphs,quotes,decorations,intersections,calc}
\usetikzlibrary{fit}
\usepackage{tikzpeople}

\usepackage[
lambda,advantage,adversary,probability,
sets,notions,asymptotics,keys,
operators,ff,primitives
]{cryptocode}

\newcolumntype{C}[1]{>{\centering\arraybackslash}m{#1}}

\usepackage{tcolorbox}
\newcommand{\say}[1]{\emph{``{#1}''}}

\usepackage{listings}
\usepackage{color}

\definecolor{mygreen}{rgb}{0,0.6,0}
\definecolor{mygray}{rgb}{0.5,0.5,0.5}
\definecolor{mymauve}{rgb}{0.58,0,0.82}

\lstset{ 
  backgroundcolor=\color{white},   % choose the background color; you must add \usepackage{color} or \usepackage{xcolor}; should come as last argument
  basicstyle=\footnotesize,        % the size of the fonts that are used for the code
  breakatwhitespace=false,         % sets if automatic breaks should only happen at whitespace
  breaklines=true,                 % sets automatic line breaking
  captionpos=b,                    % sets the caption-position to bottom
  commentstyle=\color{mygreen},    % comment style
  deletekeywords={...},            % if you want to delete keywords from the given language
  escapeinside={\%*}{*)},          % if you want to add LaTeX within your code
  extendedchars=true,              % lets you use non-ASCII characters; for 8-bits encodings only, does not work with UTF-8
  firstnumber=1,                   % start line enumeration with line 1
  frame=single,                    % adds a frame around the code
  keepspaces=true,                 % keeps spaces in text, useful for keeping indentation of code (possibly needs columns=flexible)
  keywordstyle=\color{blue},       % keyword style
  language=Octave,                 % the language of the code
  morekeywords={*,...},            % if you want to add more keywords to the set
  numbers=left,                    % where to put the line-numbers; possible values are (none, left, right)
  numbersep=5pt,                   % how far the line-numbers are from the code
  numberstyle=\tiny\color{mygray}, % the style that is used for the line-numbers
  rulecolor=\color{black},         % if not set, the frame-color may be changed on line-breaks within not-black text (e.g. comments (green here))
  showspaces=false,                % show spaces everywhere adding particular underscores; it overrides 'showstringspaces'
  showstringspaces=false,          % underline spaces within strings only
  showtabs=false,                  % show tabs within strings adding particular underscores
  stepnumber=2,                    % the step between two line-numbers. If it's 1, each line will be numbered
  stringstyle=\color{mymauve},     % string literal style
  tabsize=2,                     % sets default tabsize to 2 spaces
  title=\lstname                   % show the filename of files included with \lstinputlisting; also try caption instead of title
}


% !TeX root = ../report.tex

% General Macros
% and Notation

\newcommand{\msf}[1]{\ensuremath{\textsf{#1}}}

\NewDocumentCommand{\oneinputalg}{R(){Alg}m}{%
	\IfBlankTF{#2}{%
		\ensuremath{{#1}}
	}{%
		\ensuremath{{#1}(#2)}
	}%
}

\NewDocumentCommand{\twoinputalg}{R(){Alg}mom}{%
	\IfBlankTF{#4}{%
		\ensuremath{{#1}}
	}{%
		\IfValueTF{#3}{%
			\ensuremath{{#1}(#3,#4)}
		}{%
			\ensuremath{{#1}(#2,#4)}
		}%
	}%
}

\NewDocumentCommand{\threeinputalg}{R(){Alg}momom}{%
	\IfBlankTF{#6}{%
		\ensuremath{#1}
	}{%
		\IfValueTF{#5}{%
			\IfValueTF{#3}{%
				\ensuremath{{#1}(#3,#5,#6)}
			}{
				\ensuremath{{#1}(#2,#5,#6)}
			}%
		}{%
			\IfValueTF{#3}{%
				\ensuremath{{#1}(#3,#4,#6)}
			}{%
				\ensuremath{{#1}(#2,#4,#6)}
			}%
		}%
	}%
}

\NewDocumentCommand{\fourinputalg}{R(){Alg}momomom}{%
	\IfBlankTF{#8}{%
		\ensuremath{#1}
	}{%
		\IfValueTF{#7}{%
			\IfValueTF{#5}{%
				\IfValueTF{#3}{%
					\ensuremath{{#1}(#3,#5,#7,#8)}
				}{
					\ensuremath{{#1}(#2,#5,#7,#8)}
				}%
			}{%
				\IfValueTF{#3}{%
					\ensuremath{{#1}(#3,#4,#7,#8)}
				}{%
					\ensuremath{{#1}(#2,#4,#7,#8)}
				}%
			}%
		}{%
			\IfValueTF{#5}{%
				\IfValueTF{#3}{%
					\ensuremath{{#1}(#3,#5,#6,#8)}
				}{
					\ensuremath{{#1}(#2,#5,#6,#8)}
				}%
			}{%
				\IfValueTF{#3}{%
					\ensuremath{{#1}(#3,#4,#6,#8)}
				}{%
					\ensuremath{{#1}(#2,#4,#6,#8)}
				}%
			}%
		}%
	}%
}

\NewDocumentCommand{\optindex}{om}{%
	\IfValueTF{#1}{%
		\ensuremath{{#2}_{#1}}%
	}{%
		\ensuremath{{#2}}%
	}%
}

\newcommand{\red}[1]{\textcolor{red}{#1}}
\newcommand{\oran}[1]{\textcolor{orange}{#1}}
\newcommand{\blue}[1]{\textcolor{blue}{#1}}
\newcommand{\gray}[1]{\textcolor{gray}{#1}}


\newcommand{\ok}{\msf{ok!}}
\newcommand{\bad}{\msf{bad}}

\newcommand{\valid}{\msf{valid}}
\newcommand{\fail}{\msf{fail}}


\newcommand{\PoSW}{\msf{PoSW}}

\newcommand{\qprime}{\ensuremath{q}}
\newcommand{\qpprime}{\ensuremath{p}}
\newcommand{\gammaR}{\ensuremath{\gamma_{\Rring}}}



\NewDocumentCommand{\witness}{o}{\optindex[#1]{\omega}}
\RenewDocumentCommand{\proof}{o}{\optindex[#1]{\pi}}
\NewDocumentCommand{\transcript}{o}{\optindex[#1]{\tau}}


\renewcommand{\vec}[1]{\ensuremath{\textbf{#1}}}

\NewDocumentCommand{\vx}{o}{\optindex[#1]{\vec{x}}}
\NewDocumentCommand{\vy}{o}{\optindex[#1]{\vec{y}}}
\NewDocumentCommand{\vu}{o}{\optindex[#1]{\vec{u}}}

\NewDocumentCommand{\vA}{o}{\optindex[#1]{\vec{A}}}
\NewDocumentCommand{\vG}{o}{\optindex[#1]{\vec{G}}}
\NewDocumentCommand{\vGi}{o}{\optindex[#1]{\vec{G}}^{-1}}

\NewDocumentCommand{\Rring}{o}{\optindex[#1]{\mathcal{R}}}

\RenewDocumentCommand{\pp}{o}{\optindex[#1]{\msf{pp}}}


\NewDocumentCommand{\ffun}{om}{\oneinputalg(\optindex[#1]{\msf{f}}){#2}}

\NewDocumentCommand{\hashf}{om}{\oneinputalg(\optindex[#1]{\msf{H}}){#2}}


\newcommand{\init}{\ensuremath{\msf{Init}}}
\newcommand{\gen}{\ensuremath{\msf{Gen}}}
\renewcommand{\eval}{\ensuremath{\msf{Eval}}}
\newcommand{\prove}{\ensuremath{\msf{P}}}
\renewcommand{\verify}{\ensuremath{\msf{V}}}
\newcommand{\proveR}{\ensuremath{\overline{\msf{P}}}}
\newcommand{\verifyR}{\ensuremath{\overline{\msf{V}}}}

\newcommand{\pprime}{\ensuremath{p}}



\newcommand{\evaluate}{\ensuremath{\textsf{Eval}}}
\newcommand{\Poly}{\ensuremath{\textsf{Poly}}}

\renewcommand\adversary[1]{\ensuremath{\mathcal{#1}}}
\newcommand{\advP}{\adversary{P}}
\newcommand{\advV}{\adversary{V}}



\newcommand{\card}[1]{\ensuremath{\left\lvert #1 \right\rvert}}
\newcommand{\comp}[1]{\ensuremath{\left\lVert #1 \right\rVert}}
\renewcommand{\asymp}[1]{\ensuremath{\mathcal{O}\left(#1\right)}}

\makeatletter
\newcommand{\tpmod}[1]{\mkern 3mu({\operator@font mod}\mkern 6mu#1)}
\makeatother


\newcommand{\cbnote}[1]{\todo[inline,caption={},color=blue!20]{cB: #1}}
\newcommand{\draftnote}[1]{\todo[inline,caption={},color=green!20]{#1}}
\newcommand{\mindnote}[1]{\todo[inline,caption={},color=gray!10]{#1}}

\def\BibTeX{{\rm B\kern-.05em{\sc i\kern-.025em b}\kern-.08em
    T\kern-.1667em\lower.7ex\hbox{E}\kern-.125emX}}



\usepackage{todonotes}
\newcommand{\cbnote}[1]{\todo[inline,caption={},color=red!10]{\footnotesize{}cB: #1}}
\newcommand{\metanote}[1]{\todo[inline,caption={},color=gray!10]{\footnotesize{}#1}}
\newcommand{\probnote}[1]{\todo[inline,caption={},color=blue!10]{\footnotesize{}#1}}

\author{Carlo Brunetta}
\title{Technical Assessment}

\begin{document}

	\maketitle

	% ---
	FRI-based polynomial commitment schemes are commonly used for building transparent zkSNARKs.
	Some variants of such schemes have a wide range of parameters for finding a trade-off between
	the performance and security of the overall system.
	For example, a larger proximity parameter of the FRI protocol allows for a more efficient
	commitment scheme but requires careful consideration when analyzing the system's security level.

	We propose to consider two specific proof systems, Redshift~\cite{EPRINT:KPV19} and
	Plonky2~\cite{Plonky2} (both use the FRI-based commitment scheme), and answer the following
	research questions.

	\begin{enumerate}

		\item Redshift describes ways to guarantee the uniqueness of setup polynomials
			(see section 4.3~\cite{EPRINT:KPV19}).
			Namely, they propose to use extra evaluation points. How many extra points will ensure a
			128-bit security level for a 256-bit finite field and code rate $= \frac{1}{8}$?
		
		\item Describe (briefly) how an attacker can exploit the absence of such extra points.
		
		\item Does the Soundness notion capture such types of attacks?
		
		\item Does Plonky2 use extra evaluation points? If yes, how many are used? Why exactly so many?
			If not, is the system still safe to use? Why?
		
		\item Can the attack described in the first answer be applied to Plonky2?
	\end{enumerate}

	To prepare your answers, you may need to refer to the source code of these proof systems.
	We expect you to prepare your answer using \LaTeX.


	\section{Preliminaries}

		Let $\FF[X]$ denote the polynomial ring over the field $\FF$ and the set of functions from $D$ to $C$
		as $C^D$.
		Let $f \in \FF^D$ and denote $\interpolant{f}$ as the unique polynomial
		$\interpolant{f}(X) = \sum_{i=0}^{\card{S}-1} a_i X^i$ with degree
		$\polydeg{\interpolant{f}} < D$ with same graph as $f$, \ie\
		$\forall_{s \in S}, f(s) = \interpolant{f}(s)$.
		%
		For a set $L$, let $\zeropoly{L} = \prod_{l_i \in L} (X-l_i)$ be the unique polynomial with
		degree $\polydeg{\zeropoly{L}} = \card{L}$ that vanishes on $L$.

		Define the Reed-Solomon code of rate $\rho \in (0,1] $ evaluated over $D$ on the field $\FF$ as
		the set of functions,
		\[ \rs{\FF}{D}{\rho} = \left\{ \vphantom{\sum}f : D \to \FF \quad \middle\lvert \quad
			\polydeg{f} < \rho \cdot \card{D} \right\} \]
		To later make sense, $D$ must be an affine subspace of $\FF$ whenever $\FF = \FF_{2^m}$ while for
		$\FF = \FF_p$ for some prime $p$, $D$ is a multiplicative subgroup of $\FF_p$.
		%
		Define the Hamming distance between two polynomials $f,g: D \to \FF$ as
		$\hamdist{f}{g} = \card{ \{ a \in D \:\mid\: f(a) \neq g(a) \}}$ and the distance between
		a polynomial and a set of function as $\hamdist{f}{G} = \min_{g \in G} \hamdist{f}{g}$.
		The relative Hamming distance is obtained by normalizing the distance with respect to $\card{D}$.
		
		
		\subsection{Polynomial Commitment Scheme}
		
		A polynomial commitment scheme is a scheme that allows a prover to publicly commit to a function
		$f$ and later provide proof of evaluation correctness of $f(\alpha)$ to the input $\alpha$
		provided by a verifier.
		
		\begin{definition}\label{def:polycom}
			A polynomial commitment scheme $\proto{}$ is a tuple of algorithms:
			\begin{itemize}
				\item $\setup(\secpar,\FF,d) \to \pp$: given security parameter $\secpar$, field $\FF$ and 
					maximal degree $d$, generates the schemes public parameters $\pp$.
				\item $\pcommit(\pp,f(X)) \to \comm{f}$: given the public parameters and a polynomial $f(X)$,
					the prover outputs a commitment $\comm{f}$.
				\item $\ppolyverify(\pp,f(X), \comm{})$: given the public parameters, a polynomial $f(X)$
					and a commitment $\comm{}$, output if the polynomial is consistent with the commitment.
				\item $\popen(\pp,f(X), \alpha) \to (y, \proo{})$: given public parameters,
					a prover is asked by a verifier to provide the evaluation of $f(\alpha)$ for a verifier's
					chosen value $\alpha$.
					The prover provides the evaluation $y$ and a correctness proof $\proo{}$.
				\item $\pverify(\pp,\comm{f},\alpha,y,\proo{})$: given the public parameters, a poylnomial
					commitment $\comm{f}$, an evaluation point $\alpha$, the result of the evaluation $y$ and
					proof $\proo{}$, the verifier verifies the correctness of the evaluation $y = f(\alpha)$
					by accepting or rejecting the proof.
			\end{itemize}
		\end{definition}
		
		If the generation can be done without the involvement of a trusted third party (or similar),
		the commitment scheme is said to be \emph{transparent}.
		%
		A polynomial commitment scheme should achieve several properties, the most relevant ones are
		(intuitively described):
		
		\begin{itemize}
			\item \emph{Polynomial binding}:	the commitment should bind the polynomial meaning that once
				the prover publishes $\comm{f}$, it is difficult to find another polynomial $g$ of degree
				$\polydeg{g} \leq d$ with the same commitment.
			
			\item \emph{Evaluation binding}: it is hard for the prover to cheat the opening meaning that
				if the prover opens to an incorrect value $z \neq f(\alpha)$, with high probability the
				verifier will reject the proof.
				
			\item \emph{Polynomial hiding}: the verifier should be unable to obtain too much information
				on the polynomial from the opening (of course within limitation on the number of evaluations
				allowed).
			
			\item \emph{Succinctness}: the scheme has short proofs and computation times.
				More specifically, we care only on schemes that communication costs and computation time
				poly-logarithmic in the polynomial degree $\polydeg{f}$.
		\end{itemize}
		


		\subsection{Fast Reed-Solomon IOPP}
		
			Both \redshift\ and \plonkyy\ are based on the Fast Reed-Solomon Interactive oracle proofs
			of proximity~\cite{ICALP:BBHR18} (FRI) which is a protocol that allows a prover to 
			show that a function $f$ is $\delta$ close to a polynomial in $\rs{\FF}{L^{(0)}}{\rho}$.
			
			There are some subtle differences between the two which provide differences in efficiency
			and security~\cite[Sec.6.2]{Plonky2}:
			\redshift\ utilizes \FRI\ as originally described by Ben-Sasson et al.~\cite{ICALP:BBHR18}
			despite some improvements are already discussed in the appendix.
			
			\plonkyy\ is based on \DEEPFRI~\cite{EPRINT:BGKS19}
			which is an improved version of \FRI~\cite{ICALP:BBHR18} where an additional evaluation
			point outside the domain is used to increase the soundness guarantees.
			Additionally, \plonkyy\ mainly uses batching, \ie\ several polynomials
			$\{f_i\}_{i=0}^{\card{B}-1}$ are linearly combined using a verifier's provided
			random scalar $\lambda$ as $h(X) = \sum_{i=0}^{\card{B}-1} \lambda^i \cdot f_i(X)$
			and later the \FRI\ protocol is executed on $h(X)$ which values opening can be
			computed from opening the values of $f_i(X)$ and computing the correct
			linear combination using $\lambda$.
			%
			Furthermore, \plonkyy\ utilizes Merkle cap instead of Merkle root~\cite{EPRINT:ChiYog21},
			\ie\ a sorted list of Merkle root of the subtrees on a specified stopping level, and
			\emph{grinding} which is a forced proof-of-work that the prover must execute and improve
			the soundness guarantees.
			
			Despite the differences between the \FRI's protocol used between \plonkyy\ and \redshift,
			the rest of the document will discuss the effective differences in how the two are used
			to obtain a polynomial commitment scheme.
			
			
			

	\section{\redshift}

		Proving \FRI\ implies that the code rate $\rho$ is chosen in such a way guarantee the
		uniqueness of the closest polynomial $p \in \rs{\FF}{D}{\rho}$ such that
		$\hamdist{f}{p} < \delta$.
		%
		\redshift\ tweaks this idea by considering a list of close polynomials
		$(p_i)_{i\in I} \subseteq \rs{\FF}{D}{\rho}$ s.t. each polynomial is close, \ie\
		$\hamdist{f}{p_i} < \delta$.
		Having such a list can guarantee the protocol to prove the existence of a close
		polynomial but not necessarily the uniqueness.
		This is the core of \redshift's \emph{list polynomial commitment} idea.
		
		\subsection{Too Many Close Polynomials}
		
		However, there is a problem when trying to guarantee the evaluation correctness of the
		setup polynomial $s(X)$ from which output defines specific constrain which must be
		correct and verified and not confused with the evaluation of one of the close polynomials
		$g_i \in \rs{\FF}{D}{\rho}$.
		Technically, the verifier can compute such constraints ($s$ is known since it's a setup polynomial)
		however in a non-succinct way since they must evaluate $s(X)$ which is linear in $\polydeg{s}$.
		
		To avoid this problem, both the prover and verifier can compute the decoding list $L_{\delta}(s)$
		of polynomials $\delta$-relatively close to the setup polynomial $s(X)$ and later find a
		value $\xi$ such that $s(\xi) \neq g_i(\xi)$ for all $g_i \in L_{\delta}(s)$.
		This evaluation point is used to distinguish $s(X)$ from all the other close polynomials.
		
		\redshift's authors suggest two wait to find such a point.
		The first method computes the decoding list $L_{\delta}(s)$ which has cost $\complexity{\card{D}^3}$
		for $\delta < 1 - (d-N-\mu)$ where $D$ is the function domain, $d$ the maximal degree considered,
		$N$ number of evaluation points and $\mu$ the number of repetitions while, when considering
		equality, the complexity increases to $\complexity{\card{D}^{15}}$.
		Once the close polynomials are found, with an overhead linear in $\card{D}$, it is possible 
		to find such distinguishing point $\xi$.
		
		The second method is based on random sampling $\mu$ points and uses them as distinguishing points.
		Due to the Schwartz-Zippel lemma, with high-probability these points will indeed separate
		$s$ from the corresponding list $g \in L_{\delta}(s)$.
		This takes $\complexity{\mu\cdot d}$ which is substantially faster when $d \sim \card{D}$
		however at the cost of a reduced soundness guarantee.
		
		
		\probnote{
			Describe (briefly) how an attacker can exploit the absence of such extra points.
		}
		
		The absence of the distinguishing point(s) provides an insecurity when committing polynomials 
		and requiring them to be uniquely binded (such as setup polynomials).
		%
		Consider the setup polynomial $s(X)$ and the corresponding decoding list $L_{\delta}(s)$.
		Each close polynomial $g \in L_{\delta}(s)$ is $\delta$ close to $s(X)$ thus an adversary
		can switch the polynomial $s(X)$ with $g(X)$ which has a non-empty decoding list
		intersection $L_{\delta}(s) \cap L_{\delta}(g) \neq \emptyset$.
		%
		This implies that the adversary forces a wrongful setup polynomial $g(X)$ which might have
		(in some relevant evaluation point $\alpha$) a different value $g(\alpha) \neq s(\alpha)$.
		
		For example and intuitively, in the \redshift's instantiation of Algorithm~2~\cite{EPRINT:KPV19},
		an adversary able to force a wrong output for the setup polynomials $q_j,S_{i_j},S_{\sigma_j}$
		can force a different circuit to be calculated (by modifying the appropriate evaluation of $q_j$
		and/or how the index are permuted by $S_{i_j},S_{\sigma_j}$).
		
		
		\probnote{
			Does the Soundness notion capture such types of attacks?
		}
	
		The soundness definition does not capture such an attack because, from a relation point of view,
		the proof provided for the wrong but still $\delta$-close polynomial $g(X)$ is correct
		because generated for the intersection of the decoding lists
		$L_{\delta}(s) \cap L_{\delta}(g) \neq \emptyset$ thus allowing the existence of an element
		in the relation.
		%
		Intuitively, adding the distinguishing evaluation point adds a constraint in the relation
		that allows the unique identification of the witness, \ie\ the relation turns into
		something like ``the witness polynomial is $\delta$-close to the decoding list \textbf{and}
		the distinguishing evaluation is different from \emph{all} the decoded polynomials''.
		
		
		\subsection{Concrete Parameters}
		
		\probnote{
			Redshift describes ways to guarantee the uniqueness of setup polynomials
			(see section 4.3~\cite{EPRINT:KPV19}).
			Namely, they propose to use extra evaluation points.
			How many extra points will ensure a 128-bit security level for a 256-bit finite field
			and code rate $= \frac{1}{8}$?
		}
		
		To provide concrete parameters for the polynomial commitment scheme, let us focus on the
		results provided by Thm.~4, Thm.~5, Claim~1, Thm.~6 respectively stating the list
		polynomial commitment's soundness is the sum of the \FRI\ and the distinguisher's soundness,
		the distinguishing soundness obtained from the random sampling and how all the pieces
		come together for the polynomial commitment parameters.
		
		\[ \underbrace{ \frac{2 \log(\card{D})}{\eta^3 \card{\FF}} +
			\left(1 - \min\{\delta_0,\delta\} + \eta\log(\card{D})\right)^\mu }_{\text{\FRI\ sound.}} +
			\underbrace{ \left(\frac{d}{\card{\FF}} \cdot \left(J_{\rho,\nu} -1\right) \right)^\mu }_{\text{Dist. sound.}} \]
		
		We are interested in soundness $< 2^{-128}$, field $\log(\card{\FF}) = 256$ and code rate
		$\rho= \frac{1}{8}$.
		
		Similarly to the author's argument~\cite[Sect.6]{EPRINT:KPV19}, consider 
		$\nu = \card{\FF}^{-\frac{1}{20}}$ which provides a decoding list size of
		$\frac{\rho^{-\frac{1}{2}}}{2} \card{\FF}^{\frac{1}{20}} = J_{\rho,\nu}$.
		%
		Furthermore, $\log(\card{D}) - 3 = \log(d - N - \mu)$ and $d-N-\mu > 16$ where $d$ is the degree,
		$N$ the number of extra evaluation points and $\mu$ the number of repetitions.
		For the sake of simplicity, consider $N=1$ as in a single opening of an evaluation of the
		committed function meaning $d\sim \card{D}$.
		We consider the \FRI\ protocol to be executed for all the $\log(d)$ rounds meaning that the
		amount of verifier's queries are $l \sim 2\cdot\log(d)$ and with $\card{D} = 2^{32}$ as the
		authors discuss.
		Finally, $\delta_0 = \frac{1-\rho}{2}$ and $\delta \in (0, 1 - \sqrt{1-(1-\rho)\cdot (1-\nu)}$
		meaning $\delta_0 = \frac{7}{16}$ and $\delta \in \sim(0, \frac{10}{16})$
		
		By putting everything together, we get that,
		
		\begin{align*}
			\frac{2 \log(2^{32})}{2^{-\frac{(256) \cdot 3}{20}} \cdot 2^{256}} +
				\left(1 - \min\{\delta_0,\delta\} + 2^{-\frac{256}{20}}\log(2^{32})\right)^{\mu} +
				\left(\frac{2^{32}}{2^{256}} \cdot \left(\frac{2^{\frac{3}{2}}}{2}
				2^{\frac{256}{20}} -1\right) \right)^\mu  &< 2^{-128}\\
			\frac{2^6}{2^{217.6}} +
				\left(1 - \frac{7}{16} + 2^{-7.8}\right)^{\mu} +
				\left(\frac{2^{13.3}}{2^{224}}\right)^\mu &< 2^{-128}\\
			0.5625^\mu + 2^{-210.7 \mu}
				&\lesssim 2^{-128} - 2^{211.6}\\
				\mu &\gtrsim \frac{-128}{-0.83} \simeq 154.2
		\end{align*}

		If the committed polynomial must be binded, this implies $\mu \sim 155$ repetition of the \FRI\
		protocol implying $\mu$ additional points.
		
		The authors provide a soundness analysis for \redshift\ instantiated as a Plonk-like
		system~\cite[Sec.6]{EPRINT:KPV19}.
		Despite having similar security parameters, \ie\ they focus on $80$-bit security instead
		of $128$, the field dimensions already provide a high soundness guarantee for the distinguisher
		used in the list polynomial commitment scheme.
		However, their \FRI\ soundness estimation is $\sim 0.504$ for the same $\nu$ but $\rho = \frac{1}{16}$
		which is in line with the $\sim 0.5625$ obtained in our case with $\rho = \frac{1}{8}$.
		The computed amount $\mu$ looks coherent with the amount of repetition presented in Table~1
		considering that the table is limited to $80$-bit of security and $\rho=\frac{1}{16}$.
		
		

		
	\section{\plonkyy}
		\plonkyy\ is based on \sffont{TurboPLONK} on which provides an extensive amount of
		circuit gadget's improvements and specific parameter choices to increase the performance
		and efficiency of the protocol.
		As additional differences from \redshift, \plonkyy\ utilizes the Domain Extending
		for Eliminating Pretenders \FRI\ (\DEEPFRI)~\cite{EPRINT:BGKS19}
		for the polynomial commitment scheme and it allows recursion, \ie\ intuitively the scheme's proving
		computation can be proved by the same scheme thus allowing to
		aggregate and compress proofs to a smaller size and with higher computational efficiency.
		
		
		\subsection{Differences}
		\probnote{
			Does Plonky2 use extra evaluation points?
			If yes, how many are used?
			Why exactly so many?
			If not, is the system still safe to use?
			Why?
		}
		\probnote{
			Can the attack described in the first answer be applied to Plonky2?
		}
		
		
		To accommodate recursion, the polynomial distance $\delta$ must be larger introducing multiple
		binding polynomials thus creating the same argument as \redshift.
		Similarly to \redshift, \plonkyy\ requires an additional evaluation point used to correctly
		bind the commitment to a specific polynomial however how binding is guaranteed is different.
		
		Instead of requiring an evaluation point $\alpha$ in which the polynomial $s(X)$ is different
		from all the other $\delta$-close polynomials $g_i(X)$, \plonkyy\ requires a $\zeta$ value
		chosen from the considered extension field $\FF_p(\phi)$ (\ie\ $\FF_p[X] / (X^2 - 7)$, a
		degree $1$ polynomial basically) and all the protocols' polynomials are evaluated on such
		point~\cite[Sec. 7.2, point 6]{Plonky2}.
		%
		This is part of the \DEEP\ methodology where to discriminate between several possible
		$\delta$-close polynomials, the verifier requests the evaluation $f(\zeta)$ on a random
		element $\zeta$ outside the domain and considers the quotient regarding such evaluation.
		Intuitively, the only way for the wrong $\delta$-close polynomial $g$ to have the
		quotient $\frac{g(X) - f(\zeta)}{X-\zeta}$ to be a polynomial by either having $g=f$ 
		(an argument \emph{a-la} Schwartz-Zippel lemma) or $g$ has some
		peculiar shape which, most likely, will be inconsistent with being $\delta$-close or
		$g(\zeta) = f(\zeta)$.
		
		Additionally, the additional evaluation is verified with the oracle's commitment thus an
		adversary providing wrong evaluation will be spotted since the \FRI\ protocol is executed on the
		quotient polynomial.
		%
		For \plonkyy\, the majority of these checks are effectively executed during the \FRI's
		protocol execution, since the prover batch aggregates all the quotient polynomials to prove into
		a single linear combination using coefficients provided by the verifier.
		
	\bibliographystyle{alphaurl}
	\bibliography{biblio}

\end{document} 