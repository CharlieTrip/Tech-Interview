% !TeX root = ../report.tex

\section{Applications}\label{sec:applications}

\begin{center}
	\itshape Present one or two applications of the previous protocol and/or system.
\end{center}

The \PoSW\ primitive is one of many primitives in the
timed cryptography domain which has some well-established application classes.
The applications that utilise the \PoSW\ primitives are presented.

\vspace{2mm}
\mindnote{
	\emph{Personal Note:}
	if $T$ implies a delay $\Delta$, effectively there are other interesting application which
	not necessarily requires the verification step except for fast-forwarding between
	sequential evaluations.
	
	These applications are based on the idea of time-lock puzzles.
	For example, one can construct time-capsules, i.e. the ``encrypting for the future''
	idea~\cite{EC:BDDNO21}) or turn-based consistent communication channel~\cite{PROVSEC:BLLMT21}
	(a-la blockchain).

	The ``issue''	is that $\Delta$ time must be invested making the primitive realistically
	prohibitive to use in several applications.
	The ``alternative'' would be to have a trapdoor to quickly obtain $\vy$ which goes against
	the \PoSW\ concept.
	That is why I avoided presenting them, except for this personal note!
}



\paragraph{Lotteries and MPC Fairness.}
Guaranteeing randomness to the randomly sampled is hard and important.
The main concern is that the random oracle is an ideal object while the hash/PRF are the closest
objects that we commonly agree to look/be pseudo-random.
When considering multi-party settings, it is of vital importance that any pseudo-random evaluation
is verifiable to prevent a malicious actor from influencing the output in their favour.

If computing $\PoSW$ on some large $T$ has controlled computationally cost/timing $\Delta$
(one can somehow fix $T$ to identify a computational timing $\Delta$), then the primitive can be
used to run anything in the lottery class:
a conductor sets up a lottery where people can register before a pre-decided registration deadline.
When the deadline is over, the conductor must honestly sample a winner at random from the
registered people which can verify that the sampling was done correctly.

This can be done by letting the conductor set up a \PoSW, collect all the registration and use
such information as the input $\vx$ for the $\PoSW$.
Because of the primitive sequentiality and the full input being effectively available only
at the deadline,
if $T$ is big enough to create an appropriate computational delay $\Delta$, the final winner is
unknown to everyone (conductor comprised) making the lottery fair.
Plus, the conductor must provide a proof $\pi$ which people can use to verify the
correct winner's sampling without investing $\Delta$ time.

The class of lotteries contains many applications where the goal is to select an unpredictable
candidate from a list.
For example, one can use them for other type of consensus protocols, emulating the creation
of a common-reference-string, sampling randomness in multi-party computations.
	


\paragraph{Expiring Password/Secrets}
	
This application wants to force a prover to expire a password/secret.
Consider a shared secret, between a prover and verifier, $\vx$ with a liveliness-budget of $T$.
The verifier, starting from a secretly shared value $\vy[T]$ and $T$, requires the prover to slowly
consume its budget by an amount $t_i \leq T$ until $\vx$ is revealed.
%
This is easy to achieve if the prover creates the challenge and never releases $\vx$.
When the verifier requests to verify and consume $t_1$, the prover provides the proof
of correct $t_1$ evaluation from $\vy[T-t_1]$ until $\vy[T]$.
The next request on $t_2$ provides the proof from $\vy[T-t_1-t_2]$ until $\vy[T-t_1]$ and so on.
When $\sum t_i > T$, the prover should not be able to provide additional proofs forcing the
prover to output $\vx$ which expires the secret.

This idea is similar to a multi-stage commitment scheme where the commitments are chained and
opened in sequence.
The effective gain is that there is an additional possibility to quickly open $t_i$ stages
with a reduced cost for verifying the commitment.


\vspace{3mm}
A possible application would use the secrets $\vy[i]$ for $i\in[0,T]$ until the verifier
requests to consume the budget.
Since both parties have the secrets $\vx$ and $T$, they both can compute $\witness$ and $\vy[T]$.
Let's assume that the challenge $(\vA,\vx,T)$ is provided by the verifier to the prover
and they use $\vy[T]$ as a shared secret.

At any moment, the verifier \advV\ might get suspicious that the prover \advP\ got hacked and
lost control of the shared secret $\vy[T]$ and maybe some other intermediate value from the witness.
For this reason, \advV\ requests to consume $t$ evaluation to verify if \advP\ has enough
partial evaluations or not.
Both parties execute the verification of correct evaluation from $\vy[T-t]$ until $\vy[T]$.
If \valid, \advV\ is reassured that \advP\ has (at least) $t$ pre-images of the one-way
evaluation $\ffun{}$.

From now onwards, they will use $\vy[T-t-1]$ as a shared secret, allowing the repetition of the 
consume-request until all $T$ evaluations are consumed.
At that point, the prover is unable to provide verifying proof forcing both to lose any shared
secret between themselves.
Observe that \advV\ can quickly verify if \advP\ has the same shared secret, e.g. any symmetrically 
encrypted fixed message with a derivate key from the shared secret would work.


As an added feature, executing the budget-consume query will fully expose the last shared secret
which can be useful to execute a verification of some exchanged message, i.e. messages are
exchanged and have a \msf{MAC} computed from the secret key derived from the shared secret.
By revealing the shared secret, anyone can verify the messages' correctness.



\vspace{2mm}
\mindnote{
	I hope the description is clear.
	I think I never encountered such an application but feels somehow useful or, at least,
	something that can have an interesting real application!
}