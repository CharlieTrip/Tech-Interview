% !TeX root = ../report.tex

\section{Non-Interactive Transformation}

\begin{center}
	\itshape
	Turn the protocol into a non-interactive proof system, and write the full specification.
\end{center}


To obtain a non-interactive proof system, I would suggest using the Fiat-Shamir (FS) transformation.
In a nutshell, the prover substitutes requesting the verifier to sample and send a random challenge $c$
with the output of a random oracle (RO) query on the current communication transcript.
The key idea is that it is hard for the prover to predict the oracle's output thus
forcing the prover to follow the protocol honestly since the verifier can re-compute the
same values from the oracle too during verification.
Additionally, the verifier is never queried thus obtaining the non-interactivity.

\vspace{3mm}
\mindnote{
	I'm not sure if I will have the time later but I tried to (very) quickly check some
	zero-knowledge papers (mainly \cite{EPRINT:AteFehKlo21}) to see if there is a ``safer'' way to
	deal with this sort of ``dependent multi-round + parallel challenges'' but couldn't find
	anything pointing to a better methodology.
	
	Therefore, I will go for the solution that \emph{feels} more secure!
}

\paragraph{Sketch Idea.}

First, one must introduce a concrete instantiation for the RO.
For the sake of simplicity, consider a cryptographic hash function $\hashf{}$ with codomain $S$
which must be introduced during the instantiation of the public parameters $\pp$.

The interactive protocol is split into the procedures for proving and verifying the evaluations
which follows the protocol structure by reading/writing into a transcript $\transcript$ instead
of receiving/sending messages to the other party.
%
As the name suggests, the transcript emulates the communication between the prover and verifier
thus the prover can instantiate it with some initial description, e.g. the public parameters,
challenge and result $\transcript[0] = \pp \concat \vA \concat \vx \concat \vy$.

The prover starts executing $\prove(\vA,(\vx,\vy,\pprime,T), \witness)$ and each time
it should send/receive a message, this is appended to $\transcript[i-1]$ to get $\transcript[i]$.
In particular, the received messages are all randomly sampled elements $r_i$ which are computed
by hashing the current transcript $r_i = \hashf{\transcript[i-1]}$.
At the end of the emulated communication, the prover outputs the final transcript as the non-interactive
proof $\proof = \transcript[t_1+t_2]$.

The verifier received the proof $\pi$, starts from the same initial transcript $\transcript[0]$
and sequentially reads the messages in the transcript.
Since $T$ is known, the verifier knows which are the random elements thus it can verify the correct
execution by recomputing and verifying the correct usage of all the $r_i = \hashf{\transcript[i-1]}$.

\vspace{2mm}
\mindnote{
	Using a cryptographic hash function $\hashf{}$ as an instantiation of a random oracle
	is sometimes arguable since it might not be the best choice for some applications.
	
	Maybe a \msf{PRF} and pushing the problem of getting a common random key/seed?
	
	I'm choosing to not go (for now) into the rabbit hole and find out the most correct type of function!
}




\subsection{Non-Interactive \PoSW\ Primitive}

Let me report the protocol of \Cref{sec:prototype} and highlight in \blue{blue} the addition
made to transform the primitive into a non-interactive one.
%
For the sake of readability, the algorithms $\eval,\gen$ are not reported since they
are unaltered in the non-interactive version.
%
The algorithms are defined as follows:

\begin{itemize}
	\item $\init(\secpar) \to \pp$: initialise the public parameters $\pp$ containing 
		the security parameter $\secpar$, the ring $\Rring$ of integers with a power-of-prime
		$\poly[\secpar]$-th root of unity, values $\qprime$ and $\pprime$, dimensions $n$ and $m$,
		\oran{subtractive set $S$}, \oran{expansion value $\gammaR$}
		\blue{and secure hash function $\hashf{}$ (representing the random oracle)}.
	
	% \item $\gen(\pp) \to (\vA,\vx)$: generate and output the challenge by uniformly sampling
	% 	$\vA \sample \Rring[q]^{n \times m}$ and $\vx \sample \Rring[q]^{n}$
	
	% \item $\eval(\vA,\vx,T) \to (\vy, \witness)$: for all $i \in [0,T-1]$ with $T > 0$ and
	% 	$\vx[0] = \vx$, evaluate $\ffun{\vx[i]} = - \vA\vGi(\vx[i]) \pmod{q} = \vx[i+1]$
	% 	and store the values $\vu[i] = -\vGi(\vx[i])$.
	% 	Output the witness $\witness = (\vu[i])_{i=0}^{T-1}$ and evaluation result $\vy = \vx[T]$.
	
	\item \blue{$\prove(\vA,(\vx,\vy,\beta,T), \witness) \to \proof$}: the proving procedure is computed
		recursively by calling the procedure
		$\proveR(\vA,(\vx,\vy,\beta,T), \witness,\transcript)$
		with starting transcript $\transcript = \pp\concat\vA\concat\vx\concat\vy\concat R$
		where $R$ is a uniformly random bit-string.
		The procedure is defined as follows:
		\begin{itemize}
			\item if $T = 1$:
			\begin{itemize}
				\item \blue{the procedure} \blue{appends} $\vu[0]$ \blue{to $\transcript$}
				\item \blue{output the final transcript $\transcript$}
			\end{itemize}
			
			\item if $T = 2 \cdot t$:
			\begin{itemize}
				\item \blue{the procedure} \blue{appends} $\vu[T-1]$ \blue{to $\transcript$ and obtains $\transcript^\prime$}
				\item \blue{recursively} execute
					$\blue{\proveR}(\vA,(\vx,- \vG \vu[T-1],\beta,T-1), (\vu[i])_{i=0}^{T-2}
						\blue{,\transcript^\prime}  )$
			\end{itemize}
			
			\item if $T = 2 \cdot t + 1 > 1$:
			\begin{itemize}
				\item \blue{the procedure} \blue{appends} $\vu[t]$ \blue{to $\transcript$ and
					obtains $\transcript^\prime$}
				\item \blue{the procedure computes} $r = \blue{\hashf{\transcript^\prime}}$ \blue{and
					appends it to $\transcript^\prime$ and obtains $\transcript^{\prime\prime}$}
				\item \blue{the procedure} computes:
					\begin{itemize}
						\item $\vx^\prime = \vx + \vA\cdot\vu[t]\cdot r \pmod{\qprime}$
						\item $\vy^\prime = \vy\cdot r - \vG\cdot\vu[t] \pmod{\qprime}$
						\item $\beta^\prime = 2 \cdot \gammaR \cdot \beta$
						\item $\vu[i]^\prime = \vu[i] + r\cdot\vu[t+1+i]$ for $i \in [0,t-1] $
					\end{itemize}
					
				\item \blue{recursively} execute
					$\blue{\proveR}(\vA,(\vx^\prime,\vy^\prime,\beta^\prime,t), (\vu[i]^\prime)_{i=0}^{t-1},
						\blue{\transcript^{\prime\prime}} )$
			\end{itemize}
		\end{itemize}
		\blue{At the end of the recursive execution, the final transcript is the outputted
		proof $\proof \gets \transcript$}.
	
	
	
	\item \blue{$\verify(\vA,(\vx,\vy,\beta,T),\proof) \to \{\valid,\fail\}$}:
		the verifying procedure is computed
		recursively by calling the procedure
		$\verifyR(\vA,(\vx,\vy,\beta,T), \witness,\proof,\transcript)$
		with starting transcript $\transcript = \pp\concat\vA\concat\vx\concat\vy\concat R$
		where $R$ is a uniformly random bit-string.
		For readability, reading from the proof \emph{consumes} the proof $\proof$
		(similarly to an iterator) and
		appends the content to the transcript $\transcript$ (and adds a $\prime$ in the notation, e.g.
		from $\transcript \to \transcript^\prime \to \transcript^{\prime\prime}$).
		The procedure is defined as follows:
		\begin{itemize}
			\item if $T = 1$:
			\begin{itemize}
				\item \blue{the procedure} \blue{reads} $\vu[0]$ \blue{from $\proof$}
				\item \blue{the procedure} outputs $\ok$ if $\vu[0] \in \Rring[\beta]^{m}$,
					$\vG \vu[0] = -\vx$ and $\vA \vu[0] = \vy$. Otherwise, outputs $\bad$.
			\end{itemize}
			
			\item if $T = 2 \cdot t$:
			\begin{itemize}
				\item \blue{the procedure} \blue{reads} $\vu[T-1]$ \blue{from $\proof$}
				\item \blue{the procedure} verifies that $\vu[T-1] \in \Rring[\beta]^{m}$,
					and $\vA \vu[T-1] = \vy$.
					If not, output \bad.
				\item \blue{recursively} execute
					$\blue{\verifyR}(\vA,(\vx,- \vG \vu[T-1],\beta,T-1), 
						\blue{\proof,\transcript^\prime}  )$
			\end{itemize}
			
			\item if $T = 2 \cdot t + 1 > 1$:
			\begin{itemize}
				\item \blue{the procedure} \blue{reads} $\vu[t]$ \blue{from $\proof$}
				\item verify that $\vu[t] \in \Rring[\beta]^{m}$.
					If not, \advV\ outputs \bad.
				\item \blue{it computes} $r = \blue{\hashf{\transcript^\prime}}$ \blue{and
					checks equality by reading $r$ from $\proof$.
					If not, outputs $\bad$.}
				\item \blue{the procedure} computes:
					\begin{itemize}
						\item $\vx^\prime = \vx+\vA\cdot\vu[t]\cdot r \pmod{q}$
						\item $\vy^\prime = \vy\cdot r - \vG\cdot\vu[t] \pmod{q}$
						\item $\beta^\prime = 2 \cdot \gammaR \cdot \beta$
					\end{itemize}
				\item \blue{recursively} execute
					$\blue{\verifyR}(\vA,(\vx^\prime,\vy^\prime,\beta^\prime,t),
						\blue{\proof,\transcript^{\prime\prime}} )$
			\end{itemize}\
		\end{itemize}
		At the end of the \blue{recursive execution}, if the final output is $\ok$, return $\valid$.
		Otherwise, return $\fail$.
	
\end{itemize}

To guarantee correct soundness, the proving algorithm $\prove$ must be executed
$\frac{\secpar}{\log(\secpar)}$ times implying the necessity of randomising the starting transcript
because otherwise, since $(\pp,\vA,\vx,\vy,\witness)$ is fixed, without $R$
the algorithm $\prove$ would be deterministic thus all the proof would be the same.
%
To avoid this, the random element $R$ is introduced in the transcript.



\vspace{3mm}
\mindnote{
	Maybe the approach is somehow minimal and not that efficient
	so here some additional thoughts!
}

The total proof length $\card{\proof}$ can be computed by summing all the communication costs
presented in \Cref{sec:analysis:communication}
since the transcript comprehends the instance $(\pp,\vA,\vx)$ (better digest),
a random string $R$ and the prove-verifying communication.

To improve the proof's size, the transcript can start with the random string $R$ plus the
digest of the other elements, i.e.
$\transcript[0] = R\;\concat\;\hashf[1]{\pp\concat\vA\concat\vx\concat\vy\concat R}$, where we introduce
a different (collision-free) hash function $\hashf[1]{}$ with fixed-length binary output.
This allows the compression of the starting transcript.

Only the reported in the transcript witnesses $\vu[i]$ must be readable by the
verification procedure to be used in the different computational verifications.
All the other values (e.g. starting transcript and random values)
are computed so they can be verified in a more compact format (i.e. checking a digest).
%
Nevertheless, how the transcript is defined can have security implications thus requiring additional care.

