% !TeX root = ../report.tex

\section{Protocol and Prototype}\label{sec:prototype}

\begin{center}
\itshape
	Fully specify the resulting protocol and prototype it with your favourite programming language/tool.
\end{center}

\paragraph{Quick Notation.}
Uniform sampling is denoted as $x \sample X$.
Let $\Rring$ a ring of integer of a cyclotomic field and $f$ denotes the degree of the extension
(instead of $\phi$ used in the paper), vectors $\vec{a}$ have bold typesetting
and matrices $\vec{A}$ are uppercase.
For $\qprime \in \NN$, let $\Rring[\qprime] = \Rring/q\Rring$ and $\Rring^\times$ denote the
set of units of $\Rring$.
A set $S \subset \Rring$ is subtractive if $\forall\,{a,b \in \Rring}$ with $a \neq b$, it holds
$(a-b) \in \Rring^{\times}$.
The inverse gadget matrix $\vGi(\vx)$ returns the vector obtained by appending each $p$-ary representation
for each entry of $\vx$, i.e. for $l$ such that $q < p^l$, $x = \sum_{j=0}^{l} x_{j} p^j$ and all
$x_j < p$.
Define the gadget matrix as $\vG = I_{n} \otimes \begin{pmatrix} 1 & p & \cdots & p^l\end{pmatrix}$
which, briefly, takes a $p$-ary representation and returns the original value.

\vspace{2mm}
\mindnote{
	I highlighted in \oran{orange} two objects that were somehow not present in the protocol's description.
	However, without them, the protocol would be ill-defined/weird or unusable, mainly because either
	the value might be hard to compute or prover--verifier might not agree on the same set.\\
	I highlighted in \red{red} a typo present in the eprint version of the paper~\cite{EPRINT:LaiMal24}.
	I'm unable to get the published version to check if the typo is present there too.
}



\subsection{Lai--Malavolta \PoSW\ Protocol}

The whole \PoSW's protocol is defined as:

\begin{enumerate}
	\item initialise the public parameters $\pp \gets \init(\secpar)$
	\item generate the challenge $(\vA,\vx) \gets \gen(\pp)$
	\item provide the challenge to the prover $\advP$ which executes $\eval(\vA,\vx,T)$ for
		some amount $T$, stores the witness $\witness$ and outputs $\vy$
	\item execute $\frac{\lambda}{\log{\lambda}}$ times the verification protocol with inputs
		$\prove(\vA,(\vx,\vy,\pprime,T),\witness)$ and $\verify(\vA,(\vx,\vy,\pprime,T)$
	\item if the protocol never outputs $\bad$,
		the verification protocol outputs $\valid$.
		Otherwise, it outputs $\fail$.
\end{enumerate}

\noindent
Let me expand on the different algorithms and how they are defined.

\subsection{Construction Definition}
The primitive is defined with the following algorithms $\init, \gen, \eval$ plus an interactive
protocol between a prover and a verifier.
%
The algorithms are defined as follows:
\begin{itemize}
	\item $\init(\secpar) \to \pp$: initialise the public parameters $\pp$ containing 
		the security parameter $\secpar$, the ring $\Rring$ of integers with a power-of-prime
		$\poly[\secpar]$-th root of unity, values $q$ and $p$, dimensions $n$ and $m$,
		\oran{subtractive set $S$}, \oran{expansion value $\gammaR$}%
		% and one-way function $\hashf{}$ (representing the random oracle)
		.
	
	\item $\gen(\pp) \to (\vA,\vx)$: generate the challenge by uniformly sampling
		$\vA \sample \Rring[q]^{n \times m}$ and $\vx \sample \Rring[q]^{n}$
	
	\item $\eval(\vA,\vx,T) \to (\vy, \witness)$: for all $i \in [0,T-1]$ with $T > 0$ and
		$\vx[0] = \vx$, evaluate the function $\ffun{\vx[i]} = - \vA\vGi(\vx[i]) \pmod{q} = \vx[i+1]$
		and store the values $\vu[i] = -\vGi(\vx[i])$.
		Output the witness $\witness = (\vu[i])_{i=0}^{T-1}$ and evaluation result $\vy = \vx[T]$.
\end{itemize}

The prover $\advP$ and verifier $\advV$ are involved into a recursive-like interactive protocol
which can be defined using $\prove,\verify$ for the prover and verifier's inputs respectively.
Consider $\prove(\vA,(\vx,\vy,\beta,T), \witness)$ and $\verify(\vA,(\vx,\vy,\beta,T))$ where the
public parameters $\pp$ are implicitly input to both the parties.

\begin{itemize}
	\item if $T = 1$:
	\begin{itemize}
		\item $\advP$ sends $\vu[0]$ to $\advV$
		\item $\advV$ receives $\vu[0]$
		\item $\advV$ outputs $\ok$ if
		\[
			\vu[0] \in \Rring[\beta]^{m} \;\;\bigwedge\;\; \vG \vu[0] = -\vx ;\;\bigwedge\;\; \vA \vu[0] = \vy
		\]
			Otherwise, outputs $\bad$.
	\end{itemize}
	
	\item if $T = 2 \cdot t$:
	\begin{itemize}
		\item $\advP$ sends $\vu[T-1]$ to $\advV$
		\item $\advV$ receives $\vu[T-1]$
		\item verify that
			\[ \vu[T-1] \in \Rring[\beta]^{m} ;\;\bigwedge\;\; \vA \vu[T-1] = \vy\]
			If not, \advV\ outputs \bad.
		\item Repeat with
			\[
				\prove(\vA,(\vx,- \vG \vu[T-1],\beta,T-1), (\vu[i])_{i=0}^{T-2})
				\qquad \verify(\vA,(\vx,- \vG \vu[T-1],\beta,T-1))
			\]
	\end{itemize}
	
	\item if $T = 2 \cdot t + 1 > 1$:
	\begin{itemize}
		\item $\advP$ sends $\vu[t]$ to $\advV$
		\item $\advV$ receives $\vu[t]$
		\item verify that $\vu[t] \in \Rring[\beta]^{m}$.
			If not, \advV\ outputs \bad.
		\item $\advV$ samples $r \sample S$ and sends it to $\advP$
		\item both computes:
			\begin{itemize}
				\item $\vx^\prime = \red{+}\; \vx \;\red{+}\; \vA\cdot\vu[t]\cdot r \pmod{q}$
				\item $\vy^\prime = \vy\cdot r - \vG\cdot\vu[t] \pmod{q}$
				\item $\beta^\prime = 2 \cdot \gammaR \cdot \beta$
			\end{itemize}
		\item $\advP$ computes:
			\begin{itemize}
				\item $\vu[i]^\prime = \vu[i] + r \cdot \vu[t+1+i]$ for $i \in [0,t-1] $
			\end{itemize}
			
		\item Repeat with
			\[
				\prove(\vA,(\vx^\prime,\vy^\prime,\beta^\prime,t), (\vu[i]^\prime)_{i=0}^{t-1})
				\qquad
				\verify(\vA,(\vx^\prime,\vy^\prime,\beta^\prime,t))
			\]
	\end{itemize}
	
\end{itemize}

\vspace{5mm}

\paragraph{Typo in the Paper.}
Previously highlighted in \red{red} a fixed typo from the paper.
%
Consider the scenario $ T= 2\cdot t+1$, the paper states
$\vx^\prime = \red{-}\vx\red{-}\vA\cdot\vu[t]\cdot r \pmod{q}$.
To see why this is a typo, consider $T=3$.

After the first round of communication, the prover would compute $\vx^\prime = -\vx- r \vx[2]$,
$T^\prime=1$ and $\witness[0] = \vu[0] + r\vu[2]$.
At the next execution, the verifier $\advV$ would compute,
\[\vG\witness[0] = \vG \vu[0] + r \vu[2] =
\vG ( - \vGi(\vx)) + r \vG ( - \vGi(\vx[2]))= - \vx - r \vx[2] \neq -\vx^\prime = \vx + \vx[2]\] 
which would produce a failing verification.



\clearpage
\subsection{Prototype}

I wrote the prototype in \msf{Python} which code can be found in the \texttt{code/} folder.

\paragraph{Assumptions, Observations and Limitations.}
\begin{itemize}
	\item For the sake of simplicity, I wrote the prototype assuming the usage of the integers
		as the ring $\Rring = \ZZ$ meaning that $\gammaR = 1$ and $S = \{a,a+1\}$ for any $a \in \ZZ$.
		%
		However, the code-class of the ring $\Rring$ is abstracted (enough) to allow a
		future plug-and-play usage of a more appropriate ring.
		
	\item When transforming into $\pprime$-ary representation, $\vx$ must be converted
		back into the range $[0,q)$.
	
	\item I fixed $m = n*l$ despite the paper suggesting the possibility of using different values.
	
	\item I developed the construction as defined (i.e. vertical vectors) despite this produces
		several matrix-transpositions calls in the code which affect memory and efficiency.
\end{itemize}


\paragraph{Run the Code!}
To execute the code, one can run:
\begin{lstlisting}[language=Bash]
# Default Values

python main.py 
\end{lstlisting}
with (arbitrary) default values:
\[ (\pprime,\qprime,\;\;a,n,\;\;T,\secpar\;\;,f,m,\;\;\msf{tests}) =
	(19,101,\;\;0,10\;\;,100,1,\;\;1,1,\;\;100) \]

Otherwise, the script allows modifying the parameters using the interface,

\begin{lstlisting}[language=Bash]
usage: main.py [-h] [--p P] [--q Q] [--a A] [--n N] [--T T] [--secpar SECPAR]
               [--f F] [--m M] [--tests TESTS]

PoSW Prototype using ZZ

optional arguments:
  -h, --help       show this help message and exit
  --p P            Base of the p-ary representation
  --q Q            Modulus of R_q
  --a A            Used to create the subtractive list
  --n N            Dimension of the PoSW
  --T T            Amount of sequential computations
  --secpar SECPAR  Security parameter (not used)
  --f F            f-th root of unity (order) (not used)
  --m M            Dimension of the PoSW (not used)
  --tests TESTS    Description for tests
\end{lstlisting}

The code outputs the passing rate, i.e. number of correct prove-verification executions,
and some statistical values for both the evaluation $\evaluate$ and prove-verify $\prove+\verify$
timing.
%
For example, with the default values,
\begin{lstlisting}[language=Bash]
Passing Rate: 1.0

>>  Evaluation Time (ms)
Mean: 			10.981431007385254
Variance: 	0.0456968080243314
St. Dev: 		0.21376811741775573
Median: 		10.909080505371094

>>  Proving Time (ms)
Mean: 			3.5293102264404297
Variance: 	0.004362462959761615
St. Dev: 		0.06604894366877956
Median: 		3.5108327865600586
\end{lstlisting}

\vspace{2mm}
\mindnote{
	Observe: the proving time is smaller than the evaluation time, even for other parameters.\\
	This is coherent with \PoSW's goal!
}

